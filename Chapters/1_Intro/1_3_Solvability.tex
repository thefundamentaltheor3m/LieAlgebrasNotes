\section{Solvability and Nilpotency}

We now begin discussing some nontrivial objects in the theory of Lie algebras. Throughout this section, $L$ will denote an arbitrary Lie algebra.

\subsection{Descending Series of Ideals}

\begin{boxdefinition}[Derived Series]
    The \textbf{derived series} of $L$ is the descending series of ideals
    \begin{align*}
        L = L^{(0)} \supseteq L^{(1)} \supseteq L^{(2)} \supseteq \cdots % \supseteq L^{(n - 1)} \supseteq L^{(n)}
    \end{align*}
    where $L^{(i)} := \brac{L^{(i - 1)}, L^{(i-1)}}$ for $1 \leq i$. % \leq n$.
\end{boxdefinition}

\begin{boxdefinition}[Solvability]
    $L$ is said to be \textbf{solvable} if there exists an $n \in \N$ such that $L^{(n)} = 0$.
\end{boxdefinition}

\begin{boxdefinition}[Lower Central Series]
    The \textbf{lower central series} of $L$ is the descending series of ideals
    \begin{align*}
        L = L^{0} \supseteq L^{1} \supseteq L^{2} \supseteq \cdots % \supseteq L^{n - 1} \supseteq L^{n}
    \end{align*}
    where $L^{i} := \brac{L, L^{i-1}}$ for $1 \leq i$. % \leq n$.
\end{boxdefinition}

\begin{boxconvention}
    Elements of the derived series are denoted $L^{(i)}$, with parenthesised exponents, whereas elements of the lower central series are denoted $L^{i}$, with no parentheses.
\end{boxconvention}

\begin{boxdefinition}[Nilpotency]
    $L$ is said to be \textbf{nilpotent} if there exists an $n \in \N$ such that $L^{n} = 0$.
\end{boxdefinition}

Indeed, there is the following relationship between solvability and nilpotency.

\begin{lemma}
    For all $i \in \N$, $L^i \supseteq L^{(i)}$.
\end{lemma}
\begin{proof}
    \verb|sorry - argue by induction|
\end{proof}

\begin{boxcorollary}
    If $L$ is nilpotent, then $L$ is solvable.
\end{boxcorollary}

\subsection{Ideals, Quotients and Subalgebras}

Throughout this section, let $I \nsg L$ and $K \leq L$. Recall that $I$ is a Lie subalgebra of $L$ (cf. \Cref{Ch1:Lemma:IdealSubalg}), meaning we can impose solvability and nilpotency conditions on $I$ as well.

\begin{definition}[Solvability of Subalgebras]
    We say a subalgebra of $L$ is solvable if it is solvable as a Lie algebra in its own right.
\end{definition}

\begin{boxproposition}[Solvability Conditions]
    \hfill
    \begin{enumerate}[label = \normalfont\arabic*., noitemsep]
        \item If $L$ is solvable, then so is $\quotient{L}{I}$.
        \item If $L$ is solvable, then so is $K$.
        \item If $I$ and $\quotient{L}{I}$ are solvable, then so is $L$.
    \end{enumerate}
\end{boxproposition}
\begin{proof}
    Let $\phi : L \surj \quotient{L}{I}$ be the quotient map. \verb|sorry - check phone|
\end{proof}

We have similar results for nilpotency.

\begin{boxproposition}[Nilpotency Conditions]
    \hfill
    \begin{enumerate}[label = \normalfont\arabic*., noitemsep]
        \item If $L$ is nilpotent, then so is $\quotient{L}{I}$.
        \item If $L$ is nilpotent, then so is $K$.
    \end{enumerate}
\end{boxproposition}

We will not prove these results here, as they are very similar to the corresponding results for solvability.
