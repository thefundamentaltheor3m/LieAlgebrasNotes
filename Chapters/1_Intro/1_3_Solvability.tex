\section{Solvability and Nilpotency}

We now begin discussing some nontrivial objects in the theory of Lie algebras. Throughout this section, $L$ will denote an arbitrary Lie algebra.

\subsection{Descending Series of Ideals}

\begin{boxdefinition}[Derived Series]
    The \textbf{derived series} of $L$ is the descending series of ideals
    \begin{align*}
        L = L^{(0)} \supseteq L^{(1)} \supseteq L^{(2)} \supseteq \cdots % \supseteq L^{(n - 1)} \supseteq L^{(n)}
    \end{align*}
    where $L^{(i)} := \brac{L^{(i - 1)}, L^{(i-1)}}$ for $i \geq 1$. % \leq n$.
\end{boxdefinition}

\begin{boxdefinition}[Solvability]
    $L$ is said to be \textbf{solvable} if there exists an $n \in \N$ such that $L^{(n)} = 0$.
\end{boxdefinition}

\begin{boxdefinition}[Lower Central Series]
    The \textbf{lower central series} of $L$ is the descending series of ideals
    \begin{align*}
        L = L^{0} \supseteq L^{1} \supseteq L^{2} \supseteq \cdots % \supseteq L^{n - 1} \supseteq L^{n}
    \end{align*}
    where $L^{i} := \brac{L, L^{i-1}}$ for $i \geq 1$. % \leq n$.
\end{boxdefinition}

\begin{boxconvention}
    Elements of the derived series are denoted $L^{(i)}$, with parenthesised superscript indices, whereas elements of the lower central series are denoted $L^{i}$, with no parentheses around the indices.
\end{boxconvention}

\begin{boxdefinition}[Nilpotency]
    $L$ is said to be \textbf{nilpotent} if there exists an $n \in \N$ such that $L^{n} = 0$.
\end{boxdefinition}

Indeed, there is the following relationship between solvability and nilpotency.

\begin{lemma}\label{Ch1:Lemma:DerivedSeriesContainedInLowerCentralSeries}
    For all $i \in \N$, $L^i \supseteq L^{(i)}$.
\end{lemma}
\begin{proof}
    We argue by induction on $i$. The base case is trivial, because $L^0 = L = L^{(0)}$. Now, fix $i \in \N$ and assume that $L^i \supseteq L^{(i)}$. Then,
    \begin{align*}
        L^{i + 1}
        = \brac{L, L^i} &= \brac{L, L^{(i)}} \\
        &= \Span{\setst{\brac{\ell, x}}{x \in L^{(i)}, \ell \in L}} \\
        &\supseteq \Span{\setst{\brac{\ell, x}}{x \in L^{(i)}, \ell \in L^{(i)}}} \\
        &= \brac{L^{(i)}, L^{(i)}} = L^{(i + 1)}
    \end{align*}
    where the inclusion on the third line follows from the fact that $L^{(i)} \subseteq L$. This completes the induction and proves the desired result for all $i \in \N$.
\end{proof}

\begin{boxcorollary}
    If $L$ is nilpotent, then $L$ is solvable.
\end{boxcorollary}
\begin{proof}
    \Cref{Ch1:Lemma:DerivedSeriesContainedInLowerCentralSeries} tells us that for all $n \in \N$, $L^n = 0$ implies $L^{(n)} = 0$. Thus, if such an $n$ exists that makes $L$ nilpotent, the same $n$ would also make $L$ solvable.
\end{proof}

\subsection{Ideals, Quotients and Subalgebras}

Throughout this section, let $I \nsg L$ and $K \leq L$. Recall that $I$ is a Lie subalgebra of $L$ (cf. \Cref{Ch1:Lemma:IdealSubalg}), meaning we can impose solvability and nilpotency conditions on $I$ as well.

\begin{definition}[Solvability of Subalgebras]
    We say a subalgebra of $L$ is \textbf{solvable} if it is solvable as a Lie algebra in its own right.
\end{definition}

\begin{boxproposition}[Solvability Conditions]
    \hfill
    \begin{enumerate}[label = \normalfont\arabic*., noitemsep]
        \item If $L$ is solvable, then so is $\quotient{L}{I}$.
        \item If $L$ is solvable, then so is $K$.
        \item If $I$ and $\quotient{L}{I}$ are solvable, then so is $L$.
    \end{enumerate}
\end{boxproposition}
\begin{proof}
    Let $\phi : L \surj \quotient{L}{I}$ be the quotient homomorphism.
    \begin{enumerate}
        \item Observe that it suffices to show that $\phiof{L^{(i)}} = \phiof{L}^{(i)}$ for all $i \in \N$: if this were true, then the existence of some $n \in \N$ such that $L^{(n)} = 0$ would imply that
        \begin{align*}
            \parenth{\quotient{L}{I}}^{(n)} = \phiof{L}^{(n)} = \phiof{L^{(n)}} = \phiof{0} = 0
        \end{align*}
        making $\quotient{L}{I}$ solvable whenever $L$ is.
        
        We will now prove that $\phiof{L^{(i)}} = \phiof{L}^{(i)}$ by induction on $i$. When $i = 0$, the result is trivial: it is true that $\phiof{L} = \phiof{L}$ by reflexivity. % Lean-like?
        Now, fix $i \in \N$ and assume that $\phiof{L^{(i)}} = \phiof{L}^{(i)}$. Then,
        \begin{align*}
            \phiof{L^{(i + 1)}} = \phiof{\brac{L^{(i)}, L^{(i)}}}
            &= \phiof{\Span{\setst{\brac{x,y}}{x, y \in L^{(i)}}}} \\
            &= \Span{\phiof{\setst{\brac{x,y}}{x, y \in L^{(i)}}}} \\
            &= \Span{\setst{\brac{\phiof{x}, \phiof{y}}}{x, y \in L^{(i)}}} \\
            &= \brac{\phiof{L^{(i)}}, \phiof{L^{(i)}}} \\
            &= \brac{\phiof{L}^{(i)}, \phiof{L}^{(i)}} = \phiof{L}^{(i + 1)}
            % &= \brac{\parenth{\quotient{L}{I}}^{(i)}, \parenth{\quotient{L}{I}}^{(i)}}
            % = \parenth{\quotient{L}{I}}^{(i + 1)}
        \end{align*}
        as required.

        \item It suffices to prove that for all $i \in \N$, $K^{(i)} \subseteq L^{(i)}$: if this were true, then the existence of some $n \in \N$ such that $L^{(n)} = 0$ would imply that $K^{(n)} = 0$, making $K$ solvable whenever $L$ is.
        
        We will now prove that $K^{(i)} \subseteq L^{(i)}$ by induction on $i$. The base case is trivial, because $K^{(0)} = K \subseteq L = L^{(0)}$. Now, fix $i \in \N$ and assume that $K^{(i)} \subseteq L^{(i)}$. Then,
        \begin{align*}
            K^{(i + 1)} = \brac{K^{(i)}, K^{(i)}}
            &= \Span{\setst{\brac{x,y}}{x, y \in K^{(i)}}} \\
            &\subseteq \Span{\setst{\brac{x,y}}{x, y \in L^{(i)}}} \\
            &= \brac{L^{(i)}, L^{(i)}} = L^{(i + 1)}
        \end{align*}
        as required.

        \item Let $m \in \N$ be such that $I^{(m)} = 0$ and let $n \in \N$ be such that $\parenth{\quotient{L}{I}}^{(n)} = 0$. It suffices to prove that for all $i, j \in \N$, $\parenth{L^{(i)}}^{(j)} = L^{(i + j)}$: if this were true, then the fact that
        \begin{align*}
            \phiof{L^{(n)}} = \parenth{\quotient{L}{I}}^{(n)} = 0
        \end{align*}
        would immediately imply that $L^{(n)} \subseteq \pker{\phi} = I$, from which it would follow that $\parenth{L^{(n)}}^{(m)} = 0$, and therefore, that $L^{(n + m)} = 0$, making $L$ solvable whenever $I$ and $\quotient{L}{I}$ are.

        We will now prove that $\parenth{L^{(i)}}^{(j)} = L^{(i + j)}$ by letting $i$ be arbitrary and performing induction on $j$. The base case is trivial, because $\parenth{L^{(i)}}^{(0)} = L^{(i)}$. Now, fix $j \in \N$ and assume that $\parenth{L^{(i)}}^{(j)} = L^{(i + j)}$. Then,
        \begin{align*}
            \parenth{L^{(i)}}^{(j + 1)} = \brac{\parenth{L^{(i)}}^{(j)}, \parenth{L^{(i)}}^{(j)}} = \brac{L^{(i + j)}, L^{(i + j)}} = L^{(i + j + 1)}
        \end{align*}
        as required.
    \end{enumerate}
\end{proof}

We have similar results for nilpotency.

\begin{definition}[Nilpotency of Subalgebras]
    We say a subalgebra of $L$ is \textbf{nilpotent} if it is solvable as a Lie algebra in its own right.
\end{definition}

\begin{boxproposition}[Nilpotency Conditions]
    \hfill
    \begin{enumerate}[label = \normalfont\arabic*., noitemsep]
        \item If $L$ is nilpotent, then so is $\quotient{L}{I}$.
        \item If $L$ is nilpotent, then so is $K$.
    \end{enumerate}
\end{boxproposition}

We will not prove these results here, as they are very similar to the corresponding results for solvability. We will, however, mention that the reason why we do not have a nilpotency condition for $L$ when $I$ and $\quotient{L}{I}$ are nilpotent is that it is not, in general, true that $\parenth{L^i}^j = L^{i + j}$ for $i, j \in \N$, as we can easily see from the following counterexample.

\begin{boxcexample}
    \sorry
\end{boxcexample}
