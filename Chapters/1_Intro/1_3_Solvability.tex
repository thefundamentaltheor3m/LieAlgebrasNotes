\section{Solvability and Nilpotency}

We now begin discussing some nontrivial objects in the theory of Lie algebras. Throughout this section, $L$ will denote an arbitrary Lie algebra.

\subsection{Descending Series of Ideals}

We begin by defining the so-called \textbf{derived series} that consists of repeated derivations, that is, repeated performing of the $'$ operation on a Lie algebra (that is, repeatedly computing the derived subalgebra).

\begin{boxdefinition}[Derived Series]
    The \textbf{derived series} of $L$ is the descending series of ideals
    \begin{align*}
        L = L^{(0)} \supseteq L^{(1)} \supseteq L^{(2)} \supseteq \cdots % \supseteq L^{(n - 1)} \supseteq L^{(n)}
    \end{align*}
    where $L^{(i)} := \brac{L^{(i - 1)}, L^{(i-1)}}$ for $i \geq 1$. % \leq n$.
\end{boxdefinition}

Each $L^{(i)}$ is nothing but the derived subalgebra of $L^{(i - 1)}$.

We have a special term for Lie algebras for which the derived series stabilises at $0$.

\begin{boxdefinition}[Solvability]
    $L$ is said to be \textbf{solvable} if there exists an $n \in \N$ such that $L^{(n)} = 0$.
\end{boxdefinition}

We have already encountered a trivial family of solvable Lie algebras.

\begin{boxexample}
    Every abelian Lie algebra $L$ is solvable. Its derived series is simply
    \begin{align*}
        L = L^{(0)} \supseteq L^{(1)} = \brac{L, L} = 0
    \end{align*}
\end{boxexample}

There is also a less trivial example.

\begin{boxexample}\label{Ch1:Eg:2D_NonAbelian_Lie_Algebra_solvable}
    $\r_2$ (cf. \Cref{Ch1:Eg:2D_NonAbelian_Lie_Algebra}) is solvable. Its derived series is
    \begin{align*}
        \r_2 &= \r_2^{(0)} = \Span{
            \begin{bmatrix}
                1 & 0 \\ 0 & 1
            \end{bmatrix},
            \begin{bmatrix}
                0 & 1 \\ 0 & 0
            \end{bmatrix}
        } \\
        &\supsetneq \r_2^{(1)} = \brac{\r_2, \r_2} = \Span{
            \begin{bmatrix}
                0 & 1 \\ 0 & 0
            \end{bmatrix}
        } \\
        &\supsetneq \r_2^{(2)} = \brac{\r_2^{(1)}, \r_2^{(1)}} = 0
    \end{align*}
    because $\r_2^{(1)}$, being one-dimensional, is abelian.
\end{boxexample}

Given that we understand all Lie algebras of dimension $2$, we can make such seemingly sweeping statements as ``All Lie algebras of dimension $2$ are solvable.''

Next, we define the \textbf{lower central series} of a Lie algebra. This is a series of ideals that is similar to the derived series, but instead of repeatedly taking the derived subalgebra, we repeatedly take the commutator with the parent Lie algebra.

\begin{boxdefinition}[Lower Central Series]
    The \textbf{lower central series} of $L$ is the descending series of ideals
    \begin{align*}
        L = L^{0} \supseteq L^{1} \supseteq L^{2} \supseteq \cdots % \supseteq L^{n - 1} \supseteq L^{n}
    \end{align*}
    where $L^{i} := \brac{L, L^{i-1}}$ for $i \geq 1$. % \leq n$.
\end{boxdefinition}

The use of parentheses when describing the derived series is deliberate: notice that we have dropped the parentheses in the above definition.

\begin{boxconvention}
    Elements of the derived series are denoted $L^{(i)}$, with parenthesised superscript indices, whereas elements of the lower central series are denoted $L^{i}$, with no parentheses around the indices.
\end{boxconvention}

We give a special term to Lie algebras where this process of repeatedly taking the commutator with the parent Lie algebra stabilises at $0$.

\begin{boxdefinition}[Nilpotency]
    $L$ is said to be \textbf{nilpotent} if there exists an $n \in \N$ such that $L^{n} = 0$.
\end{boxdefinition}

The name is not accidental: in nilpotent Lie algebras, all adjoint maps are nilpotent. We will prove this, and even more interesting facts, in due course.

We have already encountered a trivial family of nilpotent Lie algebras.

\begin{boxexample}
    Every abelian Lie algebra $L$ is nilpotent. Its lower central series is simply
    \begin{align*}
        L = L^{0} \supseteq L^{1} = \brac{L, L} = 0
    \end{align*}
\end{boxexample}

There are nontrivial examples as well, but it will be easier to construct them once we develop more machinery.

There is a very important relationship between the derived series and the lower central series.

\begin{boxlemma}\label{Ch1:Lemma:DerivedSeriesContainedInLowerCentralSeries}
    For all $i \in \N$, $L^i \supseteq L^{(i)}$.
\end{boxlemma}
\begin{proof}
    We argue by induction on $i$. The base case is trivial, because $L^0 = L = L^{(0)}$. Now, fix $i \in \N$ and assume that $L^i \supseteq L^{(i)}$. Then,
    \begin{align*}
        L^{i + 1}
        = \brac{L, L^i} &= \brac{L, L^{(i)}} \\
        &= \Span{\setst{\brac{\ell, x}}{x \in L^{(i)}, \ell \in L}} \\
        &\supseteq \Span{\setst{\brac{\ell, x}}{x \in L^{(i)}, \ell \in L^{(i)}}} \\
        &= \brac{L^{(i)}, L^{(i)}} = L^{(i + 1)}
    \end{align*}
    where the inclusion on the third line follows from the fact that $L^{(i)} \subseteq L$. This completes the induction and proves the desired result for all $i \in \N$.
\end{proof}

This gives us a natural relationship between nilpotency and solvability.

\begin{boxcorollary}\label{Ch1:Cor:NilpotentImpliesSolvable}
    If $L$ is nilpotent, then $L$ is solvable.
\end{boxcorollary}
\begin{proof}
    \Cref{Ch1:Lemma:DerivedSeriesContainedInLowerCentralSeries} tells us that for all $n \in \N$, $L^n = 0$ implies $L^{(n)} = 0$. Thus, if such an $n$ exists that makes $L$ nilpotent, the same $n$ would also make $L$ solvable.
\end{proof}

The converse is not true.

\begin{boxcexample}[A Lie algebra that is solvable but not nilpotent]\label{Ch1:CEg:2D_NonAbelian_Lie_Algebra_solvable_not_nilpotent}
    In \Cref{Ch1:Eg:2D_NonAbelian_Lie_Algebra_solvable}, we saw that $\r_2$ is solvable. However, $\r_2$ is not nilpotent: its lower central series stabilises at a nonzero ideal. Explicitly,
    \begin{align*}
        \r_2 &= \r_2^{0} = \Span{
            \begin{bmatrix}
                1 & 0 \\ 0 & 1
            \end{bmatrix},
            \begin{bmatrix}
                0 & 1 \\ 0 & 0
            \end{bmatrix}
        } \\
        &\supsetneq \r_2^{1} = \brac{\r_2, \r_2} = \Span{
            \begin{bmatrix}
                0 & 1 \\ 0 & 0
            \end{bmatrix}
        } \\
        &= \r_2^{2} = \brac{\r_2, \r_2^1} = \Span{
            \brac{
                \begin{bmatrix}
                    1 & 0 \\ 0 & 0
                \end{bmatrix},
                \begin{bmatrix}
                    0 & 1 \\ 0 & 0
                \end{bmatrix}
            }, \brac{
                \begin{bmatrix}
                    0 & 1 \\ 0 & 0
                \end{bmatrix},
                \begin{bmatrix}
                    0 & 1 \\ 0 & 1
                \end{bmatrix}
            }
        } = \Span{
            \begin{bmatrix}
                0 & 1 \\ 0 & 0
            \end{bmatrix}
        } \\
        &= \r_2^{3} = \r_2^{4} = \cdots \\
        &\neq 0
    \end{align*}
\end{boxcexample}

In fact, one can construct several more counterexamples. Once we develop more machinery, we will be able to show, among other things, that every $\t{n}$ is solvable but not nilpotent.

For now, we end this subsection with a fact about the centres of nonzero, nilpotent Lie algebras. The point of the following is that the converse tells us when a nonzero Lie algebra is \textit{not} nilpotent, allowing us to construct counterexamples to the converse of \Cref{Ch1:Cor:NilpotentImpliesSolvable}.

\begin{boxlemma}\label{Ch1:Lemma:NilpotentCentreNonzero}
    If $L$ is nonzero and nilpotent, its centre is nonzero.
\end{boxlemma}
\begin{proof}
    Let $n$ be the largest natural number such that $L^n \neq 0$. That is, let $n \in \N$ be such that
    \begin{align*}
        L = L^0 \supseteq L^1 \supseteq \cdots \supseteq L^n \supsetneq L^{n+1} = 0
    \end{align*}
    We know such an $n$ exists not only because $L$ is nilpotent but also because $L \neq 0$, meaning that the lower central series cannot be trivial (ie, it consists of at least one proper inclusion---namely, that of the zero ideal with a non-zero ideal). By definition, $\brac{L, L^n} = L^{n+1} = 0$. Hence, $L^n = \Zof{L}$. And, as discussed above, $L^n \neq 0$. Therefore, $\Zof{L} \neq 0$, as required.
\end{proof}

We now develop some machinery that allows us to construct solvable and nilpotent Lie \textit{sub}algebras (that would, in particular, imply solvability and nilpotency when we view these Lie subalgebras as Lie algebras in their own right).

\subsection{Ideals, Quotients and Subalgebras}

For a subalgebra to be solvable means exactly what one would imagine.

\begin{boxdefinition}[Solvability of Subalgebras]
    We say a subalgebra of $L$ is \textbf{solvable} if it is solvable as a Lie algebra in its own right.
\end{boxdefinition}

For the remainder of this subsection, fix a Lie algebra $L$ and let $I \nsg L$ and $K \leq L$. It is interesting to explore the relationship between the solvability of $L$, $I$ and $K$.

\begin{boxproposition}[Solvability Conditions]\label{Ch1:Prop:SolvabilityConditions}
    \hfill
    \begin{enumerate}[label = \normalfont\arabic*., noitemsep]
        \item If $L$ is solvable, then so is $\quotient{L}{I}$.
        \item If $L$ is solvable, then so is $K$.
        \item If $I$ and $\quotient{L}{I}$ are solvable, then so is $L$.
    \end{enumerate}
\end{boxproposition}
\begin{proof}
    Let $\phi : L \surj \quotient{L}{I}$ be the quotient homomorphism.
    \begin{enumerate}
        \item Observe that it suffices to show that $\phiof{L^{(i)}} = \phiof{L}^{(i)}$ for all $i \in \N$: if this were true, then the existence of some $n \in \N$ such that $L^{(n)} = 0$ would imply that
        \begin{align*}
            \parenth{\quotient{L}{I}}^{(n)} = \phiof{L}^{(n)} = \phiof{L^{(n)}} = \phiof{0} = 0
        \end{align*}
        making $\quotient{L}{I}$ solvable whenever $L$ is.
        
        We will now prove that $\phiof{L^{(i)}} = \phiof{L}^{(i)}$ by induction on $i$. When $i = 0$, the result is trivial: it is true that $\phiof{L} = \phiof{L}$ by reflexivity. % Lean-like?
        Now, fix $i \in \N$ and assume that $\phiof{L^{(i)}} = \phiof{L}^{(i)}$. Then,
        \begin{align*}
            \phiof{L^{(i + 1)}} = \phiof{\brac{L^{(i)}, L^{(i)}}}
            &= \phiof{\Span{\setst{\brac{x,y}}{x, y \in L^{(i)}}}} \\
            &= \Span{\phiof{\setst{\brac{x,y}}{x, y \in L^{(i)}}}} \\
            &= \Span{\setst{\brac{\phiof{x}, \phiof{y}}}{x, y \in L^{(i)}}} \\
            &= \brac{\phiof{L^{(i)}}, \phiof{L^{(i)}}} \\
            &= \brac{\phiof{L}^{(i)}, \phiof{L}^{(i)}} = \phiof{L}^{(i + 1)}
            % &= \brac{\parenth{\quotient{L}{I}}^{(i)}, \parenth{\quotient{L}{I}}^{(i)}}
            % = \parenth{\quotient{L}{I}}^{(i + 1)}
        \end{align*}
        as required.

        \item It suffices to prove that for all $i \in \N$, $K^{(i)} \subseteq L^{(i)}$: if this were true, then the existence of some $n \in \N$ such that $L^{(n)} = 0$ would imply that $K^{(n)} = 0$, making $K$ solvable whenever $L$ is.
        
        We will now prove that $K^{(i)} \subseteq L^{(i)}$ by induction on $i$. The base case is trivial, because $K^{(0)} = K \subseteq L = L^{(0)}$. Now, fix $i \in \N$ and assume that $K^{(i)} \subseteq L^{(i)}$. Then,
        \begin{align*}
            K^{(i + 1)} = \brac{K^{(i)}, K^{(i)}}
            &= \Span{\setst{\brac{x,y}}{x, y \in K^{(i)}}} \\
            &\subseteq \Span{\setst{\brac{x,y}}{x, y \in L^{(i)}}} \\
            &= \brac{L^{(i)}, L^{(i)}} = L^{(i + 1)}
        \end{align*}
        as required.

        \item Let $m \in \N$ be such that $I^{(m)} = 0$ and let $n \in \N$ be such that $\parenth{\quotient{L}{I}}^{(n)} = 0$. It suffices to prove that for all $i, j \in \N$, $\parenth{L^{(i)}}^{(j)} = L^{(i + j)}$: if this were true, then the fact that
        \begin{align*}
            \phiof{L^{(n)}} = \parenth{\quotient{L}{I}}^{(n)} = 0
        \end{align*}
        would immediately imply that $L^{(n)} \subseteq \pker{\phi} = I$, from which it would follow that $\parenth{L^{(n)}}^{(m)} = 0$, and therefore, that $L^{(n + m)} = 0$, making $L$ solvable whenever $I$ and $\quotient{L}{I}$ are.

        We will now prove that $\parenth{L^{(i)}}^{(j)} = L^{(i + j)}$ by letting $i$ be arbitrary and performing induction on $j$. The base case is trivial, because $\parenth{L^{(i)}}^{(0)} = L^{(i)}$. Now, fix $j \in \N$ and assume that $\parenth{L^{(i)}}^{(j)} = L^{(i + j)}$. Then,
        \begin{align*}
            \parenth{L^{(i)}}^{(j + 1)} = \brac{\parenth{L^{(i)}}^{(j)}, \parenth{L^{(i)}}^{(j)}} = \brac{L^{(i + j)}, L^{(i + j)}} = L^{(i + j + 1)}
        \end{align*}
        as required.
    \end{enumerate}
\end{proof}

We have similar results for nilpotency.

\begin{boxdefinition}[Nilpotency of Subalgebras]
    We say a subalgebra of $L$ is \textbf{nilpotent} if it is solvable as a Lie algebra in its own right.
\end{boxdefinition}

As before, fix a Lie algebra $L$ and let $I \nsg L$ and $K \leq L$. Imposing nilpotency conditions on $L$ allows us to infer the same about $\quotient{L}{I}$ and $K$.

\begin{boxproposition}[Nilpotency Conditions]
    \hfill
    \begin{enumerate}[label = \normalfont\arabic*., noitemsep]
        \item If $L$ is nilpotent, then so is $\quotient{L}{I}$.
        \item If $L$ is nilpotent, then so is $K$.
    \end{enumerate}
\end{boxproposition}

We will not prove these results here, as they are very similar to the corresponding results for solvability. We will, however, mention that the reason why we do not have a nilpotency condition for $L$ when $I$ and $\quotient{L}{I}$ are nilpotent is that it is not, in general, true that $\parenth{L^i}^j = L^{i + j}$ for $i, j \in \N$. The reason why this holds in the derived series is that the derived series is a recursive definition that \textit{does not involve the base case}, meaning that $\parenth{L^{(i)}}^{(j)} = L^{\parenth{i + j}}$---that is, ``taking the derived subalgebra $i$ times and then taking it $j$ times is tantamount to taking it $i + j$ times''---is simply a consequence of ``doing a thing $i$ times and then doing the same thing $j$ times is tantamount to doing it $i + j$ times''. In the case of the lower central series, however, the fact that the recursive definition \textit{involves the base case} makes things problematic, because when we compute the $j$th lower central ideal of $L^i$, \textit{we have a different base case}: we are computing Lie brackets with respect to $L^i$ instead of $L$, meaning that we are ``doing a thing $i$ times and then doing a \textit{different} (if analogous) thing $j$ times''. We see this clearly in the following counterexample.

\begin{boxcexample}[$\r_2$, again]
    $\r_2$ has an ideal $I$ of dimension and co-dimension $1$ spanned by the matrix $E_{12}$ with a $1$ in the $12$ entry and $0$s everywhere else (we have already indirectly shown this in \Cref{Ch1:Eg:2D_NonAbelian_Lie_Algebra_solvable}, so we do not do so explicitly here). Since the dimension and co-dimension of $I$ are $1$, both $I$ and $\quotient{\r_2}{I}$ are abelian, making them nilpotent. However, $\r_2$ is not nilpotent, as we have already shown in \Cref{Ch1:CEg:2D_NonAbelian_Lie_Algebra_solvable_not_nilpotent}. \\

    As one might expect, this is connected to the above discussion. For all $i, j \in \N$, if $i \geq 1$, then $\r_2^{i+j} = I$. However, $\r_2^i = I$ as well, and $I$ is abelian, meaning that $I^j = 0$ if additionally $j \geq 1$. Therefore, for all $i, j \geq 1$, we have $\parenth{\r_2^i}^j = 0 \neq I = \r_2^{i + j}$.
\end{boxcexample}

It turns out that solvability also tells us about the derived subalgebra and codimesions. We begin with a simple observation.

\begin{boxlemma}\label{Ch1:Lemma:AllDerivsOfDerivEqTop}
    $L = L'$ if and only if $\forall i \in \N$, $L^{(i)} = L^{(1)} = L' = L$.
\end{boxlemma}
\begin{proof}
    One direction is trivial, so we do not bother to prove it. To prove that if $L$ equals its derived subalgebra then $L$ equals all subsequent derived subalgebras, we argue by induction on $i$. The base case is trivial, because $L^{(0)} = L$. Now, fix $i \in \N$ and assume that $L^{(i)} = L$. Then,
    \begin{align*}
        L^{(i + 1)} = \brac{L^{(i)}, L^{(i)}} = \brac{L, L} = L'
    \end{align*}
    Furthermore, it is clear that $L^{(1)} = L'$, and, by assumption, $L' = L$. This completes the induction and proves the desired result for all $i \in \N$.
\end{proof}

There is an immediate (and somewhat trivial) consequence.

\begin{boxcorollary}\label{Ch1:Cor:DerivedSubalgLtOfSolvable}
    If $L \neq 0$, then $L$ is solvable if and only if $L' < L$.
\end{boxcorollary}
\begin{proof}
    We know, from \Cref{Ch1:Lemma:AllDerivsOfDerivEqTop}, that $L' = L$ if and only if $L^{(i)} = L$ for all $i \in \N$. In particular, since $L$ is nonzero, none of the $L^{(i)}$ can be zero, which is true if and only if $L$ is not solvable. Therefore, $L$ is solvable if and only if $L' \neq L$, which is equivalent to $L' < L$.
\end{proof}

There is also a less immediate consequence that comes from applying a combination of \Cref{Ch1:Cor:DerivedSubalgLtOfSolvable} and the Correspondence Theorem (\Cref{SP:Thm:Correspondence}) to the ideals of quotient spaces of solvable Lie algebras.

\begin{boxcorollary}\label{Ch1:Cor:ExistsIdealCodim1}
If $L \neq 0$ and $L$ is solvable, there exists an ideal $I \nsg L$ of codimension $1$.
\end{boxcorollary}
\begin{proof}
    Consider the quotient Lie algebra $K := \quotient{L}{L'}$. We know that $0 < K$, because $K = 0$ would imply that $L = L'$, which is impossible because $L$ is solvable, as shown in \Cref{Ch1:Cor:DerivedSubalgLtOfSolvable}. Therefore, $K$ contains a subspace $W$ of dimension $1$. Since $K$ is abelian, \Cref{Ch1:Prop:SubspaceIdealOfAbelian} tells us that $W$ is an ideal of $K$. The Correspondence Theorem (\Cref{SP:Thm:Correspondence}) then tells us that the preimage $V$ of $W$ under the quotient epimorphism is an ideal of $L$ that contains $L'$. Simple arithmetic and dimension results from linear algebra then tell us
    \begin{align*}
        \pdim{V} = \pdim{W} + \pdim{L'} = \parenth{\pdim{L} - \pdim{L'} - 1} + \pdim{L'} = \pdim{L} - 1
    \end{align*}
\end{proof}

% Is the converse true? I wanna say it is... surely if such an I exists then L / I is an ideal of L / L'...

\subsection{The Radical Ideal}

Throughout this subsection, we will assume that $L$ is finite-dimensional.

We begin with a basic result about the sums of ideals.

\begin{boxlemma}\label{Ch1:Lemma:DerivedSeries_sum_contained_sum_derivedSeries}
    Let $I, J \nsg L$. For all $k \in \N$, we have
    \begin{align*}
        \parenth{I + J}^{(k)} \subseteq I^{(k)} + J^{(k)}
    \end{align*}
\end{boxlemma}
\begin{proof}
    We argue by induction on $k$. The base case is trivial, because $(I + J)^{(0)} = I + J = I^{(0)} + J^{(0)}$. Now, fix $k \in \N$ and assume that $(I + J)^{(k)} \subseteq I^{(k)} + J^{(k)}$. Then,
    \begin{align*}
        (I + J)^{(k + 1)} &= \brac{(I + J)^{(k)}, (I + J)^{(k)}} \\
        &\subseteq \brac{I^{(k)} + J^{(k)}, I^{(k)} + J^{(k)}} \\
        &\subseteq \brac{I^{(k)}, I^{(k)}} + \brac{I^{(k)}, J^{(k)}} + \brac{J^{(k)}, I^{(k)}} + \brac{J^{(k)}, J^{(k)}} \\
        &\subseteq I^{(k + 1)} + J^{(k + 1)}
    \end{align*}
    as required.
\end{proof}

\begin{boxlemma}\label{Ch1:Lemma:SumIdealSolvable}
    Let $I, J \nsg L$. If $I$ and $J$ are solvable, then so is $I + J \nsg L$.
\end{boxlemma}
\begin{proof}
    Let $n \in \N$ be such that $I^{(n)} = 0$ and let $m \in \N$ be such that $J^{(m)} = 0$. Since $m + n \geq m, n$ and $I^{(n)} = J^{(m)} = 0$, we know that $I^{(n + m)} = J^{(n + m)} = 0$. Then, applying \Cref{Ch1:Lemma:DerivedSeries_sum_contained_sum_derivedSeries}, we have
    \begin{align*}
        (I + J)^{(n + m)} \subseteq I^{(n + m)} + J^{(n + m)} = 0 + 0 = 0
    \end{align*}
    proving that the derived series of $I + J$ eventually stabilises at $0$. Therefore, $I + J$ is solvable.
    % Apply 2nd iso thm
\end{proof}

\begin{boxcorollary}\label{Ch1:Cor:RadExists}
    There exists a solvable ideal of $L$ that contains all other solvable ideals of $L$.
\end{boxcorollary}
\begin{proof}
    Let $R$ be a solvable ideal of $L$ of maximal dimension.\footnote{When we say maximal dimension, we mean that the dimension of $R$ is the largest possible dimension such that a solvable ideal of that dimension exists. This is well-defined because $L$ is finite-dimensional, and the dimension of any ideal of $L$ is necessarily $\leq \pdim{L}$.} Now, fix any $I \nsg L$. \Cref{Ch1:Lemma:SumIdealSolvable} tells us that $I + R$ is a solvable ideal of $L$. But, since $R$ is of maximal dimension, we know that $\pdim{I + R} \leq \pdim{R}$. Therefore, we must have that $I + R = R$. Then, we must have that $I \subseteq R$, as any $i \in I$ is expressible as $i + 0$, and since $0 \in R$, we have $i = i + 0 \in I + R = R$. Hence, $I \subseteq R$, proving that $R$ contains any ideal of $L$ that is solvable.
\end{proof}

This solvable ideal has a name.

\begin{boxdefinition}[Radical Ideal]
    The \textbf{radical ideal} of $L$ is the solvable ideal of $L$ that contains all other solvable ideals of $L$, which we know exists from \Cref{Ch1:Cor:RadExists}.
\end{boxdefinition}

Given that we frame simplicity in terms of normal subgroups in group theory, and given that their analogue in the theory of Lie algebras is ideal, we will define semi-simplicity to be a weak form of simplicity, where the non-trivial ideals that are not allowed to exist are \textit{semi-simple}.

\begin{boxdefinition}[Semi-Simplicity]\label{Ch1:Def:SemiSimple}
    We say that $L$ is \textbf{semi-simple} if its radical ideal is the $0$ ideal.
\end{boxdefinition}

It is clear, from the definition of the radical ideal, that semi-simple Lie algebras are precisely those that contain no nontrivial solvable ideals (and are not solvable themselves).

Our aim for this module will be to classify all semi-simple Lie algebras over $\C$. We will do this by first classifying all solvable Lie algebras and then using that classification to classify all semi-simple Lie algebras. We will need a \textit{lot} more machinery before we can do this, but we will get there eventually.

While we have already alluded to the following definition, we will mention it explicitly for completeness. It is no different from what we would expect in groups.

\begin{boxdefinition}[Simplicity]
    We say that $L$ is \textbf{simple} if it has no nontrivial ideals.
\end{boxdefinition}

As we have mentioned, semi-simple Lie algebras contain no nontrivial \textit{solvable} ideals (and are not solvable themselves). Simple Lie algebras contain no nontrivial ideals of \textit{any} kind, solvable or otherwise. Hence, semi-simple Lie algebras are simple, and semi-simplicity is a strictly weaker notion than simplicity.\footnote{Some authors require that a simple Lie algebra have dimension $\geq 1$, thereby excluding the $0$ Lie algebra from being simple. Under this definition, $0$ would be semi-simple but not simple, unless we also required the dimension to be at least $1$ in the definition of semi-simplicity. These are minutia, which would be very important if we were to study Lie algebras in a very formal setting, but there is no need to consider these cases in this module.}

\subsection{Ascending Series of Ideals}

We will end by talking about ascending series of ideals and their corresponding quotients. Throughout this subsection, $L$ will denote an arbitrary Lie algebra.

\begin{boxdefinition}[Ascending Central Series]
    We say that an increasing chain of ideals
    \begin{align*}
        0 \subseteq L_1 \subseteq L_2 \subseteq L_3 \subseteq \cdots
    \end{align*}
    is an \textbf{ascending central series} of $L$ if $L_1 = \Zof{L}$ and for all $i \in \N$, we have
    \begin{enumerate}
        \item $L_i \nsg L$ with quotient map $g_i : L \surj \quotient{L}{L_i}$
        \item $L_{i + 1} = g_i\inv\!\parenth{\Zof{\quotient{L}{L_i}}}$
    \end{enumerate}
\end{boxdefinition}

\begin{boxconvention}
    We will use subscripted $L_i$s to denote elements of the ascending central series, in contrast to superscripts used for the descending central series.
\end{boxconvention}

We now have an equivalent criterion for nilpotency.

\begin{boxproposition}
    $L$ is nilpotent if and only if $L_n = L$ for some $n \in \N$.
\end{boxproposition}
\begin{proof} \hfill
    \begin{description}
        \item[$\parenth{\implies}$]
            One can show by induction on $n$ that if $L^n = 0$, then $L_n = L$. Then, if $\exists n \in \N$ such that $L^n = 0$, ie, if $L$ is nilpotent, then $L_n = 0$ as well.  \sorry % (Same n works for both)

        \item[$\parenth{\impliedby}$] 
            % Use fact that image of centre in quotient map is centre of quotient
            % The idea is to show that $L^{k + 1} = \brac{L, L^k} \subseteq L_{n + k - 1}$ and $\brac{L, L^k} \subseteq \brac{L, L_{n - l}}$. And $L_{n - k}$ is the preimage of $\Zof{\quotient{L}{L_{n - k - 1}}}$
            \sorry
    \end{description}
\end{proof}

We end with another counterexample that shows that solvable Lie algebras need not be nilpotent.

\begin{boxcexample}
    For all $n$, $\t{n}$ is solvable but not nilpotent.
    \begin{proof}[Proof that $\t{n}$ is solvable]
        First, observe that $\brac{\t{n}, \t{n}} = \t{n}' = \u{n}$. By \sorry, we know that $\u{n}$ is nilpotent. Therefore, $\u{n}$ is solvable. Furthermore, $\quotient{\t{n}}{\t{n}'}$ is abelian, making it solvable by \sorry. Therefore, by \Cref{Ch1:Prop:SolvabilityConditions}, $\t{n}$ is solvable.
    \end{proof}

    \begin{proof}[Proof that $\t{n}$ is not nilpotent]
        
    \end{proof}
\end{boxcexample}
