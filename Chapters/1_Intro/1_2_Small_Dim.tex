\section{Lie Algebras of Dimension $\leq 3$}

It turns out that we do not need any particularly sophisticated machinery to classify \underline{all} Lie algebras of dimension less than or equal to $3$.

\subsection{Abelian Lie Algebras and Lie Algebras of Dimension $1$}

We begin with a simple observation about abelian Lie algebras.

\begin{boxproposition}\label{Ch1:Prop:Abelian_Lie_Algebras_Iso}
    Fix $n \in \N$. Then, any abelian Lie algebra of dimension $n$ is isomorphic to $\C^n$ with the zero bracket.
\end{boxproposition}
\begin{proof}
    Let $L$ be a Lie algebra of dimension $n$. We know there exists a $\C$-linear isomorphism $\phi : L \to \C^n$. It follows immediately that for any $x, y \in L$,
    \begin{align*}
        \phiof{\brac{x, y}} = \phiof{0} = 0 = \brac{\phiof{x}, \phiof{y}}
    \end{align*}
    A similar argument will show that $\phi\inv : \C^n \to L$, viewed as a linear map, is a Lie algebra homomorphism as well, proving that $L \cong \C^n$.
\end{proof}

The classification of Lie algebras in $1$ dimension is then straightforward. We will begin by a rather strong but straightforward result on one-dimensional subspaces of Lie algebras.

\begin{proposition}\label{Ch1:Prop:1D_Lie_Subalgebras}
    Let $L$ be a Lie algebra. Any $1$-dimensional subspace of $L$ is an abelian Lie subalgebra.
\end{proposition}
\begin{proof}
    Let $K$ be a sub-vector space of dimension $1$. We know any $\C$-basis of $K$ consists of a single, nonzero element. Consider such a basis element $x$. For any $y_1, y_2 \in L$, there exist $\lambda_1, \lambda_2 \in \C$ such that $y_1 = \lambda_1 x$ and $y_2 = \lambda_2 x$. Then,
    \begin{align*}
        \brac{y_1, y_2} = \brac{\lambda_1 x_1, \lambda_1 x_2} = \lambda_1 \lambda_2 \brac{x, x} = 0
    \end{align*}
    proving that $\liebrac = 0$. Since $K$ is a subspace, $0 \in K$, proving that $K$ is a Lie subalgebra.
\end{proof}

The classification of Lie algebras of dimension $1$ is then immediate.

\begin{corollary}
    Any Lie algebra of dimension $1$ is abelian, isomorphic to $\C$ equipped with the zero bracket.
\end{corollary}
\begin{proof}
    Let $L$ be a Lie algebra of dimension $1$. That $L$ is abelian follows from applying \Cref{Ch1:Prop:1D_Lie_Subalgebras} to $L$ viewed as a subspace of itself. The isomorphism then follows immediately from \Cref{Ch1:Prop:Abelian_Lie_Algebras_Iso}.
\end{proof}

We can now turn our attention to the slightly more non-trivial problem of classifying non-abelian Lie algebras of dimension $2$ and $3$.

\subsection{Lie Algebras of Dimension $2$}

From \Cref{Ch1:Prop:Abelian_Lie_Algebras_Iso}, we already know that there is only one abelian Lie algebra of dimension $2$. The question remains, how many non-abelian Lie algebras of dimension $2$ are there?

We begin by giving an example.

\begin{boxexample}[A Two-Dimensional Non-Abelian Lie Algebra]
    Consider the set
    \begin{align*}
        L := \setst{
            \begin{bmatrix}
                a & b \\ 0 & 0
            \end{bmatrix}
        }{a, b \in \C}
        = \Span{\begin{bmatrix} 1 & 0 \\ 0 & 0 \end{bmatrix}, \begin{bmatrix} 0 & 1 \\ 0 & 0 \end{bmatrix}}
        \subseteq \gl{2}
    \end{align*}
    Clearly, $L$ is a linear subspace of $\gl{2}$. Furthermore, One can show that
    \begin{align*}
        \brac{
            \begin{bmatrix} 1 & 0 \\ 0 & 0 \end{bmatrix}, \begin{bmatrix} 0 & 1 \\ 0 & 0 \end{bmatrix}
        } = \begin{bmatrix} 0 & 1 \\ 0 & 0 \end{bmatrix}
    \end{align*}
    proving that $L$ is closed under the commutator bracket. It follows that $L$ is a Lie subalgebra of $\gl{2}$, and therefore, a $2$-dimensional Lie algebra in its own right.
\end{boxexample}

It turns out that we are done!

\begin{boxproposition}
    Any non-abelian Lie algebra of dimension $2$ is isomorphic to $L$.
\end{boxproposition}
\begin{proof}
    Let $K$ be a non-abelian Lie algebra of dimension $2$. It suffices to show that $K$ admits a basis $\set{x, y}$ such that $\brac{x, y} = y$, as this will readily yield the right structure constants.\footnote{Alternatively, if we can show that $\brac{x, y} - y$, it will follow immediately that the linear isomorphism sending $x$ to $\begin{bmatrix} 1 & 0 \\ 0 & 0 \end{bmatrix}$ and $y$ to $\begin{bmatrix} 0 & 1 \\ 0 & 0 \end{bmatrix}$ is, indeed, a Lie algebra isomorphism.}
    
    Let $\set{u, v}$ be an arbitrary $\C$-basis of $K$. Then, since $K$ is non-abelian, $x := \brac{u, v} \neq 0$. Consider the $1$-dimensional the linear subspace $X := \Span{x}$. We know there exists some $z \in K \setminus \set{0}$ that is linearly independent of $x$. Write $z = \lambda u + \mu v$ for some $\lambda, \mu \in \C$. Then,
    \begin{align*}
        \brac{x, z}
        &= \brac{\brac{u, v}, \lambda u + \mu v} \\
        &= \lambda \brac{\brac{u, v}, u} + \mu \brac{\brac{u, v}, v} \\
        &= \lambda \brac{u, \brac{v, u}} - \mu \brac{v, \brac{u, v}} \\
        &= \lambda\parenth{\brac{u, \brac{v, u}} + \brac{v, \brac{u, v}}} - \parenth{\lambda + \mu}\brac{v, \brac{u, v}}
    \end{align*}
    \verb|sorry|
\end{proof}

\subsection{Lie Algebras of Dimension $3$}
