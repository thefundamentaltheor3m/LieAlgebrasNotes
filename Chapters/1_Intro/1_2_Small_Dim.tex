\section{Lie Algebras of Dimension $\leq 3$}

It turns out that we do not need any particularly sophisticated machinery to classify \underline{all} Lie algebras of dimension less than or equal to $3$.

\subsection{Abelian Lie Algebras and Lie Algebras of Dimension $1$}

We begin with a simple observation about abelian Lie algebras.

\begin{boxproposition}\label{Ch1:Prop:Abelian_Lie_Algebras_Iso}
    Fix $n \in \N$. Then, any abelian Lie algebra of dimension $n$ is isomorphic to $\C^n$ with the zero bracket.
\end{boxproposition}
\begin{proof}
    Let $L$ be a Lie algebra of dimension $n$. We know there exists a $\C$-linear isomorphism $\phi : L \to \C^n$. It follows immediately that for any $x, y \in L$,
    \begin{align*}
        \phiof{\brac{x, y}} = \phiof{0} = 0 = \brac{\phiof{x}, \phiof{y}}
    \end{align*}
    A similar argument will show that $\phi\inv : \C^n \to L$, viewed as a linear map, is a Lie algebra homomorphism as well, proving that $L \cong \C^n$.
\end{proof}

The classification of Lie algebras in $1$ dimension is then straightforward. We will begin by a rather strong but straightforward result on one-dimensional subspaces of Lie algebras.

\begin{boxproposition}\label{Ch1:Prop:1D_Lie_Subalgebras}
    Let $L$ be a Lie algebra. Any $1$-dimensional subspace of $L$ is an abelian Lie subalgebra.
\end{boxproposition}
\begin{proof}
    Let $K$ be a sub-vector space of dimension $1$. We know any $\C$-basis of $K$ consists of a single, nonzero element. Consider such a basis element $x$. For any $y_1, y_2 \in L$, there exist $\lambda_1, \lambda_2 \in \C$ such that $y_1 = \lambda_1 x$ and $y_2 = \lambda_2 x$. Then,
    \begin{align*}
        \brac{y_1, y_2} = \brac{\lambda_1 x_1, \lambda_1 x_2} = \lambda_1 \lambda_2 \brac{x, x} = 0
    \end{align*}
    proving that $\liebrac = 0$. Since $K$ is a subspace, $0 \in K$, proving that $K$ is a Lie subalgebra.
\end{proof}

The classification of Lie algebras of dimension $1$ is then immediate.

\begin{boxcorollary}\label{Ch1:Cor:1D_Lie_Algebra_Classification}
    Any Lie algebra of dimension $1$ is abelian, isomorphic to $\C$ equipped with the zero bracket.
\end{boxcorollary}
\begin{proof}
    Let $L$ be a Lie algebra of dimension $1$. That $L$ is abelian follows from applying \Cref{Ch1:Prop:1D_Lie_Subalgebras} to $L$ viewed as a subspace of itself. The isomorphism then follows immediately from \Cref{Ch1:Prop:Abelian_Lie_Algebras_Iso}.
\end{proof}

We can now turn our attention to the slightly more non-trivial problem of classifying non-abelian Lie algebras of dimension $2$ and $3$.

\subsection{Lie Algebras of Dimension $2$}\label{Ch1:Subsec:LieAlgDim2}

From \Cref{Ch1:Prop:Abelian_Lie_Algebras_Iso}, we already know that there is only one abelian Lie algebra of dimension $2$. The question remains, how many non-abelian Lie algebras of dimension $2$ are there?

We begin by giving an example.

\begin{boxexample}[A Two-Dimensional Non-Abelian Lie Algebra]\label{Ch1:Eg:2D_NonAbelian_Lie_Algebra}
    Consider the set
    \begin{align*}
        \r_2 := \setst{
            \begin{bmatrix}
                a & b \\ 0 & 0
            \end{bmatrix}
        }{a, b \in \C}
        = \Span{\begin{bmatrix} 1 & 0 \\ 0 & 0 \end{bmatrix}, \begin{bmatrix} 0 & 1 \\ 0 & 0 \end{bmatrix}}
        \subseteq \gl{2}
    \end{align*}
    Clearly, $\r_2$ is a linear subspace of $\gl{2}$. Furthermore, One can show that
    \begin{align*}
        \brac{
            \begin{bmatrix} 1 & 0 \\ 0 & 0 \end{bmatrix}, \begin{bmatrix} 0 & 1 \\ 0 & 0 \end{bmatrix}
        } = \begin{bmatrix} 0 & 1 \\ 0 & 0 \end{bmatrix}
    \end{align*}
    proving that $\r_2$ is closed under the commutator bracket. It follows that $\r_2$ is a Lie subalgebra of $\gl{2}$, and therefore, a $2$-dimensional Lie algebra in its own right.
\end{boxexample}

% The \r_2 notation comes from French literature: the \r stands for resoluble (solvable), and it is clear that the above Lie algebra is solvable.

The reason we are interested in the above example will become clear, and we will reserve the notation $\r_2$ for this particular Lie algebra. For the remainder of this section, denote by $L$ an arbitrary non-abelian Lie subalgebra of dimension $2$.

We will begin by describing the derived subalgebra $L'$ (cf. \Cref{Ch1:Def:DerivedSubalg}) of $L$.

\begin{boxlemma}
    For any $\C$-basis $\set{u, v}$ of $L$, we have that $L' = \Span{\brac{u, v}}$.
\end{boxlemma}
\begin{proof}
    Let $\set{u, v}$ be a basis of $L$. Define $x := \brac{u, v}$. Since $L$ is non-abelian, $x \neq 0$, making $X := \Span{x}$ a $1$-dimensional subspace of $L$. Seeing as $L' = \brac{L, L} = \Span{\setst{\brac{x, y}}{x, y \in L}}$, it is clear that $L' \supseteq X$. It remains to show that $L' \subseteq X$.

    It suffices to show that $\setst{\brac{x, y}}{x, y \in L} \subseteq X$. To that end, fix $a, b \in L$. We know there exist $\lambda_1, \mu_1, \lambda_2, \mu_2 \in \C$ such that $a = \lambda_1 u + \mu_1 v$ and $b = \lambda_2 u + \mu_2 v$. Then,
    \begin{align*}
        \brac{a, b}
        &= \brac{\lambda_1 u + \mu_1 v, \lambda_2 u + \mu_2 v} \\
        &= \lambda_1 \lambda_2 \underbrace{\brac{u, u}}_{= 0} + \lambda_1 \mu_2 \brac{u, v} + \mu_1 \lambda_2 \brac{v, u} + \mu_1 \mu_2 \underbrace{\brac{v, v}}_{= 0} \\
        &= \parenth{\lambda_1 \mu_2 - \mu_1 \lambda_2} \brac{u, v} \in X
    \end{align*}
    as required.
\end{proof}

This tells us, in particular, that the span of the commutator of any basis of $L$ is an ideal. We now have everything we need to describe $L$.

\begin{boxproposition}
    $L$ is isomorphic to $\r_2$.
\end{boxproposition}
\begin{proof}
    It suffices to show that $L$ admits a basis $\set{x, y}$ such that $\brac{x, y} = y$, as this will readily yield the right structure constants.\footnote{Alternatively, if we can show that $\brac{x, y} = y$, it will follow immediately that the linear isomorphism sending $x$ to $\begin{bmatrix} 1 & 0 \\ 0 & 0 \end{bmatrix}$ and $y$ to $\begin{bmatrix} 0 & 1 \\ 0 & 0 \end{bmatrix}$ is, indeed, a Lie algebra isomorphism.}
    
    Let $\set{u, v}$ be an arbitrary $\C$-basis of $L$. Let $y := \brac{u, v}$. Since $L$ is non-abelian, $y \neq 0$. Therefore, there exists some $z \in L \setminus \set{0}$ that is linearly independent of $y$. Since $\Span{y} = L' \nsg L$, we know that $\brac{z, y} \in L'$. In particular, $\exists \lambda \in \C$ such that $\brac{z, y} = \lambda y$. Furthermore, since $y$ and $z$ are linearly independent and $L$ is non-abelian, $\lambda \neq 0$. So, define $x := \lambda\inv z$. Then, $x$ is still linearly independent of $y$, making $\set{x, y}$ a basis of $L$, and $\brac{x, y} = y$, as required.
\end{proof}

Yes, it's true! Up to isomorphism, there is \underline{only one} non-abelian Lie algebra of dimension $2$. Therefore, there are \underline{only two} Lie algebras of dimension $2$: one non-abelian one and one abelian one.

We can now turn our attention to the classification of Lie algebras in dimension $3$.

\subsection{Lie Algebras of Dimension $3$}

We already know the isomorphism class of all abelian Lie algebras of dimension $3$ from \Cref{Ch1:Prop:Abelian_Lie_Algebras_Iso}. The question remains, how many non-abelian Lie algebras of dimension $3$ are there?

We begin with a simple observation. Let $L$ be a Lie algebra of dimension $3$. Assume that $L$ is non-abelian. Then, we know that $\Zof{L} \neq L$ and that $L' = \brac{L, L} \neq 0$. We can then conclude that $\pdim{\Zof{L}} \in \set{0, 1, 2}$ and $\pdim{L'} \in \set{1, 2, 3}$. The strategy we use to determine the isomorphism class of $L$ will be to consider all possible dimensions of $L'$ and use our understanding of $\Zof{L}$ to distinguish between isomorphism classes given a dimension of $L'$.

\subsubsection{\underline{The Case where $\pdim{L'} = 1$.}}

We have two possibilities: either $L' \subseteq \Zof{L}$ or $L' \not\subseteq \Zof{L}$. We will consider both cases separately.

\begin{boxexample}[The Heisenberg Lie Algebra of Dimension $3$]\label{Ch1:Eg:Heisenberg_Lie_Algebra}
    Define the \textbf{Heisenberg Lie Algebra} $\u{n}$ (for $n \in \N$) to be all matrices of the form
    \begin{align*}
        \begin{bmatrix}
            0 & & * \\
            \vdots & \ddots & \\
            0 & \cdots & 0
        \end{bmatrix}
    \end{align*}
    That is, $\u{n}$ is the set of all upper-triangular matrices in $\gl{n}$ with zeroes on the diagonal. One can apply \Cref{Ch1:Prop:Subalg_Commbracket} to show that $\u{n}$ is a Lie subalgebra of $\gl{n}$ because it is an associative subalgebra of $\gl{n}$, which one can show by showing it is an associative subalgebra of $\t{n}$ (the diagonal of a product of upper-triangular matrices is the componentwise product of the diagonals, so when the diagonals are zero, so is that of the product). \\

    In the specific case of $n = 3$, if $L = \u{3}$, then we can show that $\pdim{L'} = 1$ and $L' \subseteq \Zof{L}$. \\
    
    To show that $\pdim{L'} = 1$, we can compute the commutator of two matrices in $\u{3}$. Define
    \begin{align*}
        X = \begin{bmatrix}
            0 & a & b \\
            0 & 0 & c \\
            0 & 0 & 0
        \end{bmatrix}, \quad Y = \begin{bmatrix}
            0 & d & e \\
            0 & 0 & f \\
            0 & 0 & 0
        \end{bmatrix}
    \end{align*}
    Clearly, $X, Y \in \u{3}$. Then, computing their commutator,
    \begin{align*}
        \brac{X, Y} &= \begin{bmatrix}
            0 & a & b \\
            0 & 0 & c \\
            0 & 0 & 0
        \end{bmatrix} \begin{bmatrix}
            0 & d & e \\
            0 & 0 & f \\
            0 & 0 & 0
        \end{bmatrix} - \begin{bmatrix}
            0 & d & e \\
            0 & 0 & f \\
            0 & 0 & 0
        \end{bmatrix} \begin{bmatrix}
            0 & a & b \\
            0 & 0 & c \\
            0 & 0 & 0
        \end{bmatrix} \\
        &= \begin{bmatrix}
            0 & 0 & af \\
            0 & 0 & 0 \\
            0 & 0 & 0
        \end{bmatrix} - \begin{bmatrix}
            0 & 0 & dc \\
            0 & 0 & 0 \\
            0 & 0 & 0
        \end{bmatrix} \\
        &= \begin{bmatrix}
            0 & 0 & af - dc \\
            0 & 0 & 0 \\
            0 & 0 & 0
        \end{bmatrix}
    \end{align*}
    The span of every such commutator is easily seen to be the one-dimensional subspace spanned by the matrix $E_{13}$ with a one in the $13$-position and zeroes everywhere else (in particular, $E_{13}$ is the commutator of two matrices). Therefore, $\pdim{L'} = 1$. \\

    Since we have computed $L'$ explicitly, to show that $L' \subseteq \Zof{L}$, we only need to show that $E_{13}$ commutes with every element of $L$. Indeed, for an arbitrary matrix $X \in \u{3}$ as defined above, we have
    \begin{align*}
        \brac{X, E_{13}} = \begin{bmatrix}
            0 & a & b \\
            0 & 0 & c \\
            0 & 0 & 0
        \end{bmatrix} \begin{bmatrix}
            0 & 0 & 1 \\
            0 & 0 & 0 \\
            0 & 0 & 0
        \end{bmatrix} - \begin{bmatrix}
            0 & 0 & 1 \\
            0 & 0 & 0 \\
            0 & 0 & 0
        \end{bmatrix} \begin{bmatrix}
            0 & a & b \\
            0 & 0 & c \\
            0 & 0 & 0
        \end{bmatrix}
        = 0 - 0 = 0
    \end{align*}
    proving that $L' \subseteq \Zof{L}$.
\end{boxexample}



\sorry
