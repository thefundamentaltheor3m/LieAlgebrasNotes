\section{Lie Algebras of Dimension $\leq 3$}

It turns out that we do not need any particularly sophisticated machinery to classify \underline{all} Lie algebras of dimension less than or equal to $3$.

\subsection{Preliminaries}

We begin with a simple observation about abelian Lie algebras.

\begin{proposition}\label{Ch1:Prop:Abelian_Lie_Algebras_Iso}
    Fix $n \in \N$. Then, any abelian Lie algebra of dimension $n$ is isomorphic to $\C^n$ with the zero bracket.
\end{proposition}
\begin{proof}
    Let $L$ be a Lie algebra of dimension $n$. We know there exists a $\C$-linear isomorphism $\phi : L \to \C^n$. It follows immediately that for any $x, y \in L$,
    \begin{align*}
        \phiof{\brac{x, y}} = \phiof{0} = 0 = \brac{\phiof{x}, \phiof{y}}
    \end{align*}
    A similar argument will show that $\phi\inv : \C^n \to L$, viewed as a linear map, is a Lie algebra homomorphism as well, proving that $L \cong \C^n$.
\end{proof}

Recall that the only $\C$-vector space of dimension $1$ is $\C$ itself. \Cref{Ch1:Prop:Abelian_Lie_Algebras_Iso} then tells us that any Lie algebra of dimension $1$ is isomorphic to $\C$ with the zero bracket.

It therefore suffices to classify non-abelian Lie algebras of dimension $2$ and $3$.

\subsection{Dimension $2$}

From \Cref{Ch1:Prop:Abelian_Lie_Algebras_Iso}, we already know that there is only one abelian Lie algebra of dimension $2$. The question remains, how many non-abelian Lie algebras of dimension $2$ are there?

We begin by giving an example.

\begin{boxexample}[A Two-Dimensional Non-Abelian Lie Algebra]
    Consider the set
    \begin{align*}
        L := \setst{
            \begin{bmatrix}
                a & b \\ 0 & 0
            \end{bmatrix}
        }{a, b \in \C}
        = \Span{\begin{bmatrix} 1 & 0 \\ 0 & 0 \end{bmatrix}, \begin{bmatrix} 0 & 1 \\ 0 & 0 \end{bmatrix}}
        \subseteq \gl{2}
    \end{align*}
    Clearly, $L$ is a linear subspace of $\gl{2}$. Furthermore, One can show that
    \begin{align*}
        \brac{
            \begin{bmatrix} 1 & 0 \\ 0 & 0 \end{bmatrix}, \begin{bmatrix} 0 & 1 \\ 0 & 0 \end{bmatrix}
        } = \begin{bmatrix} 0 & 1 \\ 0 & 0 \end{bmatrix}
    \end{align*}
    proving that $L$ is closed under the commutator bracket. It follows that $L$ is a Lie subalgebra of $\gl{2}$, and therefore, a $2$-dimensional Lie algebra in its own right.
\end{boxexample}

It turns out that we are done!

\begin{boxproposition}
    Any non-abelian Lie algebra of dimension $2$ is isomorphic to $L$.
\end{boxproposition}
\begin{proof}
    Let $K$ be a Lie algebra of dimension $2$. Let $\set{u, v}$ be a $\C$-basis of $K$. Then, since $K$ is non-abelian, $x := \brac{u, v} \neq 0$.

    Since $x$ is a nonzero element of the $2$-dimensional vector space $K$ the linear subspace $K' := \Span{\brac{u, v}}$. 
\end{proof}
