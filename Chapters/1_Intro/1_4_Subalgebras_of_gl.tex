\section{Subalgebras of $\gl{n}$}

We now turn our attention to the structure of subalgebras of $\gl{n}$ for some fixed $n \in \N$. We will begin by developing some mroe general theory, following which we will prove important theorems about the structure of such subalgebras.

\subsection{Linear Algebraic and Lie Algebraic Nilpotency}

We begin by recalling a basic definition from linear algebra.

\begin{boxdefinition}[Nilpotency of Elements]
    We say that $x \in \gl{n}$ is nilpotent if there exists an $m \in \N$ such that $x^m = 0$. 
\end{boxdefinition}

We can extend this to sub-vector spaces.

\begin{boxdefinition}[Nilpotency of Subspaces]
    We say $N \leq \gl{n}$ is \textbf{nilpotent} if every element of $N$ is nilpotent.
\end{boxdefinition}

We now prove some basic results linking some linear algebraic and Lie theoretic ideas.

\begin{lemma}  % Put this elsewhere!!!!!!!!
    Let $L$ be a solvable Lie algebra. Then, there exists an ideal $M \nsg L$ whose codimension is $1$.
\end{lemma}

\begin{boxtheorem}[Engel's Theorem]
    Let $L$ be a Lie subalgebra of $\gl{n}$. If $L$ is nilpotent as a sub-vector space of $\gl{n}$, then there exists a basis of $\C^n$ with respect to which every element of $L$ is upper-triangular.
\end{boxtheorem}

Before proving Engel's Theorem, we will state and prove the following Corollary that underscores the significance of this result.

\begin{boxcorollary}
    Any nilpotent sub-vector space of $\gl{n}$ is also nilpotent as a Lie subalgebra.
\end{boxcorollary}
\begin{proof}
    \sorry
\end{proof}

For the remainder of this subsection, we will focus on proving Engel's Theorem. The high-level idea is to perform induction on $\pdim{L}$ and draw a parallel with the proof of the Jordan Canonical Form theorem\footnote{Remember, we are working over $\C$.}. The proof is rather involved, and we will split it up into several steps, which we will put together at the end of this subsection.

\begin{lemma}
    Engel's Theorem holds if for all nilpotent $N \subseteq \gl{n}$, 
\end{lemma}

\subsection{Weights of Lie Algebras}

\subsection{Lie's Theorem}

