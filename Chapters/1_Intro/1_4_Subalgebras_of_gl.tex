\section{Subalgebras of $\gl{n}$}

We now turn our attention to the structure of subalgebras of $\gl{n}$ for some fixed $n \in \N$. We will begin by developing some mroe general theory, following which we will prove important theorems about the structure of such subalgebras.

\subsection{Linear Algebraic and Lie Algebraic Nilpotency}

We begin by recalling a basic definition from linear algebra.

\begin{boxdefinition}[Nilpotency of Elements]
    We say that $x \in \gl{n}$ is \textbf{nilpotent} if there exists an $m \in \N$ such that $x^m = 0$. 
\end{boxdefinition}

We can extend this to sub-vector spaces.

\begin{boxdefinition}[Nilpotency of Subspaces]
    We say a sub-vector space $N \leq \gl{n}$ is \textbf{nilpotent} if every element of $N$ is nilpotent.
\end{boxdefinition}

We have an important relationship between linear algebraic and lie algebraic nilpotency of a Lie subalgebra.

\begin{boxtheorem}[Engel's Theorem]\label{Ch1:Thm:Engel}
    Let $N$ be a Lie subalgebra of $\gl{n}$. If $N$ is nilpotent as a sub-vector space of $\gl{n}$, then there exists a basis of $\C^n$ with respect to which every element of $N$ is upper-triangular.
\end{boxtheorem}

Before proving Engel's Theorem, we will state and prove the following Corollary that underscores the significance of this result.

\begin{boxcorollary}\label{Ch1:Cor:EngelNilpotency}
    Any nilpotent sub-vector space of $\gl{n}$ is also nilpotent as a Lie subalgebra.
\end{boxcorollary}
\begin{proof}
    \sorry
\end{proof}

For the remainder of this subsection, we will focus on proving Engel's Theorem. We will fix a nilpotent subspace $N \leq \gl{n}$. The high-level idea is to perform induction on $\pdim{L}$ and draw a parallel with the proof of the Jordan Canonical Form theorem\footnote{Remember, we are working over $\C$.}. The proof is rather involved, and we will split it up into several steps, which we will put together at the end of this subsection.

We will begin by reducing the problem to one of showing that all elements of $N$ have a common eigenvector whose eigenvalue is $0$. For the remainder of this subsection, we will denote the \textbf{simultaneous kernel} of all elements of $N$ by
\begin{align}
    N_0 := \setst{v \in \C^n}{\forall T \in N, \ \Tv = 0} = \bigcap_{T \in N} \pker{T}
    \label{Ch1:Eq:Simultaneous_Kernel_Def}
\end{align}
Note that $N_0$ is $N$-invariant: for all $T \in N$ and $v \in N_0$, we have $\Tv = 0 \in N_0$. There is therefore a natural action of $N$ on $\quotient{\C^n}{N_0}$: to any $T \in N$, we can associate the linear map
\begin{align}
    \Tbar : \quotient{\C^n}{N_0} \to \quotient{\C^n}{N_0} : v + N_0 \mapsto \Tv + N_0 \in \gl{\quotient{\C^n}{N_0}}
\end{align}

\begin{lemma}
    The map $T \mapsto \Tbar : N \to \gl{\quotient{\C^n}{N_0}}$ is a Lie algebra homomorphism.
\end{lemma}
\begin{proof}
    \sorry
    % The idea is to show it to be a homomorphism of associative algebras instead.
\end{proof}

We are now ready to reduce the proof of Engel's Theorem to showing that all the elements of $T$ have a common eigenvector with eigenvalue $0$---or, equivalently, to showing that $N_0$ is nonzero.

\begin{lemma}
    If $N_0$ is nonempty, then there exists a basis of $\C^n$ with respect to which every element of $N$ is upper-triangular.
\end{lemma}
\begin{proof}
    \sorry
    % Watch from minute 45 of the recording: https://imperial.cloud.panopto.eu/Panopto/Pages/Viewer.aspx?id=9d823173-f757-41e5-9a0e-b1fe00ff933f.
\end{proof}

\subsection{Weights of Lie Algebras}

\subsection{Lie's Theorem}

