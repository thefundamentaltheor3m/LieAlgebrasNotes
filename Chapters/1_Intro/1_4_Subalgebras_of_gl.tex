\section{Subalgebras of $\gl{n}$}

We now turn our attention to the structure of subalgebras of $\gl{n}$ for some fixed $n \in \N$. We will begin by developing some mroe general theory, following which we will prove important theorems about the structure of such subalgebras.

\subsection{Linear Algebraic and Lie Algebraic Nilpotency}

We begin by recalling a basic definition from linear algebra.

\begin{boxdefinition}[Nilpotency of Elements]
    We say that $x \in \gl{n}$ is \textbf{nilpotent} if there exists an $m \in \N$ such that $x^m = 0$. 
\end{boxdefinition}

We can extend this to sub-vector spaces.

\begin{boxdefinition}[Nilpotency of Subspaces]
    We say a sub-vector space $N \leq \gl{n}$ is \textbf{nilpotent} if every element of $N$ is nilpotent.
\end{boxdefinition}

We can say something about the adjoint of a nilpotent element.

\begin{lemma}\label{Ch1:Lemma:adNilpotnetOfNilpotent}
    Let $x \in \gl{n}$ be nilpotent. Then, $\pad{x} \in \gl{\gl{n}}$ is nilpotent as well.
\end{lemma}
\begin{proof}
    We need to show that there exists an $m \in \N$ such that the map we get by successively composing the adjoint map $\pad{x}$ $m$ times is identically zero.
    
    Fix $y \in \gl{n}$. Then,
    \begin{align*}
        \pad{x}\!(y) = \brac{x, y} &= xy - yx \\
        \pad{x}^2\!(y) = \brac{x, \brac{x, y}} &= x\brac{x, y} - \brac{x, y}x = x^2y - xyx - xyx + yx^2 \\
        \pad{x}^3\!(y) = \brac{x, \brac{x, \brac{x, y}}} &= x^3y + \cdots + xyx^2 - yx^3
    \end{align*}
    More generally, one can show that
    \begin{align*}
        \pad{x}^m\!(y) = \sum_{i = 0}^{m} \lambda_{i, m} x^i y x^{m - i}
    \end{align*}
    for all $m \in \N$ and some $\lambda_{i, m} \in \Z$. In particular, since all powers of $x$ beyond some $m$ are zero, we have that $\pad{x}^m\!(y) = 0$ for all $y \in \gl{n}$.
\end{proof}

We have an important relationship between linear algebraic and lie algebraic nilpotency of a Lie subalgebra.

\begin{boxtheorem}[Engel's Theorem]\label{Ch1:Thm:Engel}
    Let $N$ be a Lie subalgebra of $\gl{n}$. If $N$ is nilpotent as a sub-vector space of $\gl{n}$, then there exists a basis of $\C^n$ with respect to which every element of $N$ is upper-triangular.
\end{boxtheorem}

Before proving Engel's Theorem, we will state and prove the following Corollary that underscores the significance of this result.

\begin{boxcorollary}\label{Ch1:Cor:EngelNilpotency}
    Any nilpotent sub-vector space of $\gl{n}$ is also nilpotent as a Lie subalgebra.
\end{boxcorollary}
\begin{proof}
    Let $N$ be a nilpotent sub-vector space of $\gl{n}$. By Engel's Theorem, there exists a basis of $\C^n$ with respect to which every element of $N$ is upper-triangular. In particular, they must all have zeros on the diagonal, because they are nilpotent: they are of the form
    \begin{align*}
        \begin{bmatrix}
            0 & & * \\
            \vdots & \ddots & \\
            0 & \cdots & 0
        \end{bmatrix}
    \end{align*}
    \sorry
\end{proof}

For the remainder of this subsection, we will focus on proving Engel's Theorem. We will fix a nilpotent subspace $N \leq \gl{n}$. The high-level idea is to perform induction on $\pdim{L}$ and draw a parallel with the proof of the Jordan Canonical Form theorem\footnote{Remember, we are working over $\C$.}. The proof is rather involved, and we will split it up into several steps, which we will put together at the end of this subsection.

We will begin by reducing the problem to one of showing that all elements of $N$ have a common eigenvector whose eigenvalue is $0$. For the remainder of this subsection, we will denote the \textbf{simultaneous kernel} of all elements of $N$ by
\begin{align}
    U_n := \setst{v \in \C^n}{\forall T \in N, \ \Tv = 0} = \bigcap_{T \in N} \pker{T}
    \label{Ch1:Eq:Simultaneous_Kernel_Def}
\end{align}
As an intersection of sub-vector spaces, $U_n$ is a subspace of $\C^n$. Furthermore, $U_n$ is $N$-invariant: for all $T \in N$ and $v \in U_n$, we have $\Tv = 0 \in U_n$. There is therefore a natural action of $N$ on the well-defined quotient space $\quotient{\C^n}{U_n}$: to any $T \in N$, we can associate the linear map
\begin{align}
    \Tbar : \quotient{\C^n}{U_n} \to \quotient{\C^n}{U_n} : v + U_n \mapsto \Tv + U_n \in \gl{\quotient{\C^n}{U_n}}
\end{align}

\begin{lemma}
    The map $T \mapsto \Tbar : N \to \gl{\quotient{\C^n}{U_n}}$ is a Lie algebra homomorphism.
\end{lemma}
\begin{proof}
    \sorry
    % The idea is to show it to be a homomorphism of associative algebras instead.
\end{proof}

We are now ready to reduce the proof of Engel's Theorem to showing that all the elements of $T$ have a common eigenvector with eigenvalue $0$---or, equivalently, to showing that $U_n$ is nonzero.

\begin{lemma}
    If $U_n$ is nonempty, then there exists a basis of $\C^n$ with respect to which every element of $N$ is upper-triangular.
\end{lemma}
\begin{proof}
    We argue by induction on $n$. When $n = 1$, the result is vacuously true. %(?)
\end{proof}

% What exactly does this mean? Put it elsewhere perhaps...

The way we will prove Engel's Theorem is to construct a sequence of subspaces
\begin{align*}
    0 = V_0 \subsetneq V_1 \subsetneq \cdots \subsetneq V_m = \C^n
\end{align*}
such that $N(V_i) \subseteq V_{i - 1}$. We will refine this sequence so that $m = n$, ie, so that
\begin{align*}
    \pdim{\quotient{V_{i}}{V_{i - 1}}} = 1
\end{align*}
using \Cref{Ch1:Prop:ExistsIdealCodim1}. We can then take distinguished elements from each of the quotients to form a basis of $\C^n$, and this will be the basis with respect to which every element of $N$ is upper-triangular.

We will now show that $U_n$ is, indeed, nonzero.

\begin{lemma}
    There exists a nonzero vector $v \in U_n$.
\end{lemma}
\begin{proof}
    We need to show that $\Tv = 0$ for all $T \in N$. We argue by induction on $\pdim{N}$. The base case is trivial. So, let $N$ be such that for all Lie subalgebras of $\gl{n}$ of dimension less than $\pdim{N}$, \sorry
    
    Let $A \subseteq N$ be a maximal\footnote{with respect to inclusion}, proper Lie subalgebra of $N$. Consider the map
    \begin{align*}
        \phi : A \to \gl{\quotient{L}{A}} : \brac{g, \Tbar} \mapsto \overline{\brac{g, T}}
    \end{align*}
    where the map $\brac{g, T}$ is the map $gT - Tg$. One can show that this map is well-defined (\sorry) and that it is a Lie algebra homomorphism (\sorry).

    Observe that since $\pdim{\phiof{A}} \leq \pdim{A}$ and $A < L$ by the assumption that $A$ is proper, we know that $\pdim{A} < L$. Therefore,we can apply the induction hypothesis to $A$. 
    \sorry
\end{proof}

% IDEA: Begin by taking a nonzero element of the overarching Lie algebra N. We know that its bracket with itself is zero. Its span is a subalgebra because any 1D subspace is a subalgebra. We need other guys such that their brackets with the first guy are contained in their span. This is how we build the basis we need.

There is also a more general formulation of Engel's Theorem over arbitrary Lie algebras.

\begin{boxtheorem}[Engel's Theorem, Second Version]\label{Ch1:Thm:EngelOverAnyLie}
    Let $L$ be an arbitrary Lie algebra. Then, $L$ is nilpotent if and only if for all $x \in L$, the adjoint map $\pad{x}$ is nilpotent.
\end{boxtheorem}
\begin{proof}
    Assume that $L$ is nilpotent. Then, there exists some $m \in \N$ such that $L^m = 0$. Then, any composition of Lie brackets of length $m$ is zero: for all $x, y \in L$,
    \begin{align*}
        \pad{x}^n\!(y) = \brac{x, \brac{x, \cdots \brac{x, y}\cdots}} = 0
    \end{align*}
    This gives us one direction of the proof.

    For the converse direction, we apply \Cref{Ch1:Thm:Engel} (the standard formulation of Engel's Theorem) to the 
\end{proof}

\subsection{Weights of Lie Algebras}

\subsection{Lie's Theorem}

