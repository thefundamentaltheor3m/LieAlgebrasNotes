\section{Important Definitions and First Examples}

We will begin by defining the fundamental objects of study in this course. We will then provide some examples of these objects and discuss means of constructing them.

\subsection{Algebras}

We begin by recalling the notion of a bilinear map.

\begin{definition}[Bilinear Map]
    Let $V$ and $W$ be vector spaces. We say that a map $f: V \times W \to \C$ is \textbf{bilinear} if it is linear in each argument. That is, for all $v, v' \in V$, $w, w' \in W$ and $\lambda \in \C$, we have
    \begin{align*}
        \fof{v + v', w} = \fof{v, w} + \fof{v', w} \\
        \fof{v, w + w'} = \fof{v, w} + \fof{v, w'} \\
        \fof{\lambda v, w} = \lambda \fof{v, w} = \fof{v, \lambda w}
    \end{align*}
\end{definition}

We will be particularly interested in bilinear maps from a vector space to itself.

\begin{boxdefinition}[Algebra]
    An \textbf{algebra} is a vector space $A$ equipped with a bilinear map $\cdot: A \times A \to A$.
\end{boxdefinition}

\begin{boxconvention}
    Given any algebra $A$, we will often refer to the corresponding bilinear map $\cdot$ as the \textbf{multiplication} map of $A$, and denote $\cdot(x, y)$ as simply $x \cdot y$ or even $xy$ (where the definition of $\cdot$ is clear from the context) for any $x, y \in A$.
\end{boxconvention}

There are many different kinds of algebras. We will be particularly interested in Lie algebras and associative algebras.

\begin{boxdefinition}[Associative Algebras]
    We say that an algebra $A$ is \textbf{associative} if the multiplication map $\cdot$ is associative. That is, for all $x, y, z \in A$, we have
    \begin{align*}
        (x \cdot y) \cdot z = x \cdot (y \cdot z)
    \end{align*}
\end{boxdefinition}

We have all seen associative algebras before.

\begin{boxexample}[The Matrix Algebra]\label{Ch1:Eg:MatrixAlgebra}
    The set $\MnC$ of $n \times n$ matrices over $\C$ forms an associative algebra under matrix multiplication, known as the Matrix Algebra.
\end{boxexample}

We will come back to associative algebras soon enough. We will now define the main object of study in this module.

\begin{boxdefinition}[Lie Algebras]
    A \textbf{Lie algebra} is an algebra $L$ whose bilinear map $\brac{\cdot, \cdot}: L \times L \to L$ satisfies the following properties:
    \begin{enumerate}
        \item For all $x \in L$, we have $[x, x] = 0$.
        \item For all $x, y, z \in L$, we have
        \begin{align}
            \brac{x, \brac{y, z}} + \brac{y, \brac{z, x}} + \brac{z, \brac{x, y}} = 0
            \label{SP:eq:JacobiIdentity}
        \end{align}
    \end{enumerate}
    Such a bilinear map $\liebrac$ is known as a \textbf{Lie Bracket}, and~\eqref{Ch1:eq:JacobiIdentity} is known as the \textbf{Jacobi Identity}.
\end{boxdefinition}

One may recall that the $\liebrac$ notation is often used in group theory to denote the \textbf{commutator} of two elements. The reason why the same notation is used for the Lie bracket is the following.

\begin{lemma}\label{Ch1:Lemma:CommBracket}
    Let $A$ be an associative algebra. Then, the commutator map $\brac{x, y} = xy - yx$ is a Lie bracket on $A$.
\end{lemma}
\begin{proof}  % Generated by Copilot
    Clearly, $\brac{x, x} = xx - xx = 0$ for all $x \in A$. We now show that $\liebrac$ satisfies~\eqref{SP:eq:JacobiIdentity}: for all $x, y, z \in A$, we have
    \begin{align*}
        \brac{x, \brac{y, z}} + \brac{y, \brac{z, x}} + \brac{z, \brac{x, y}} &= \brac{x, yz - zy} + \brac{y, zx - xz} + \brac{z, xy - yx} \\
        &= % x(yz - zy) - (yz - zy)x + y(zx - xz) - (zx - xz)y + z(xy - yx) - (xy - yx)z \\
        % &= xyz - xzy - yzx + yxz + yzx - yxz - zxy + zyx + zxy - zyx - xyz + yxz \\
        6xyz - 6xyz = 0
    \end{align*}
    where we skip over some of the intermediate computations because they are tedious and uninteresting.
\end{proof}

\Cref{Ch1:Lemma:CommBracket} gives us a large class of examples of Lie algebras. One of the most important of these is the following.

\begin{boxexample}[General Linear Lie Algebra]\label{Ch1:Eg:gl}
    For all $n \in \N$, the set of all $n \times n$ matrices forms a Lie algebra under the commutator bracket: this follows immediately from applying \Cref{Ch1:Lemma:CommBracket} to \Cref{Ch1:Eg:MatrixAlgebra}. We call this the \textbf{General Linear Lie Algebra}, denoted $\gl{n}$.
\end{boxexample}

\begin{boxconvention}
    We will denote by $\MnC$ the set of all $n \times n$ matrices, viewed (interchangeably) as a \textit{set}, a \textit{vector space} or an \textit{associative algebra}. When viewing it as a \textit{Lie algebra under the commutator bracket}, we will adopt the notation $\gl{n, \C}$, where $\C$ can be replaced by any field. We will usually abbreviate this to $\gl{n}$, because we will primarily work over $\C$.
\end{boxconvention}

Lastly, we will define the notion of an abelian Lie algebra.

\begin{definition}[Abelian Lie Algebra]
    A Lie algebra $A$ is said to be \textbf{abelian} if for all $x \in A$, we have $\brac{x, x} = 0$.
\end{definition}

The reason for this terminology is that if $A$ is an associative algebra whose multiplication map is commutative, then its commutator bracket is identically zero, making the corresponding Lie algebra abelian.

\begin{boxexample}\label{Ch1:Eg:gl1}
    Clearly, $\gl{1}$ is abelian: for all $x, y \in \gl{1} = \C$, we have $xy - yx = 0$.
\end{boxexample}

We will now define subalgebras and homomorphisms of algebras, which will allow us to construct more examples of algebras (Lie and otherwise).

\subsection{Subalgebras and Homomorphisms}

As with objects in any category, we have subobjects and morphisms. We will define these over general algebras and apply them to get more examples of Lie algebras.

\begin{boxdefinition}[Subalgebras]
    Let $A$ be a vector space. A \textbf{subalgebra} of $A$ is a subspace $B \subseteq A$ such that $B$ is closed under the multiplication map of $A$. That is, for all $x, y \in B$, we have $x \cdot y \in B$.
\end{boxdefinition}

\begin{boxconvention}
    Given an algebra $A$ and a subset $B \subseteq A$, we will denote the statement that $B$ is a subalgebra of $A$ by $B \leq A$.
\end{boxconvention}

\begin{boxdefinition}[Homomorphisms]
    Let $A$ and $B$ be algebras. A \textbf{homomorphism} $\phi: A \to B$ is a linear map that respects the multiplication maps of $A$ and $B$. That is, for all $x, y \in A$, we have
    \begin{align*}
        \phiof{x \cdot y} = \phiof{x} \cdot \phiof{y}
    \end{align*}
\end{boxdefinition}

\begin{boxconvention}
    We will define Lie subalgebras to be subalgebras with respect to the algebra structure given by the Lie bracket, and we will define Lie algebra homomorphisms to be homomorphisms that respect the Lie bracket (ie, that are algebra homomorphisms with respect to the algebra structure given by the Lie bracket).
\end{boxconvention}

We have the following unsurprising result.

\begin{lemma}\label{Ch1:Lemma:im_ker_subalg}
    Let $A$ and $B$ be algebras, and let $\phi: A \to B$ be a homomorphism. Then,
    \begin{enumerate}[label= \normalfont\arabic*., noitemsep]
        \item $\pim{\phi} \leq B$
        \item $\pker{\phi} \leq A$
    \end{enumerate}
\end{lemma}
\begin{proof}
    These are standard results, but we will prove them for completentess.
    % AI-generated proof with minor human edits
    \begin{enumerate}
        \item Fix $x, y \in \pim{\phi}$. Then, there exist $a, b \in A$ such that $\phiof{a} = x$ and $\phiof{b} = y$. Since $\phi$ is a homomorphism, we have
        \begin{align*}
            x \cdot y = \phiof{a} \cdot \phiof{b} = \phiof{a \cdot b} \in \pim{\phi}
        \end{align*}
        so $\pim{\phi}$ is closed under the multiplication map of $B$.
        \item Let $x, y \in \pker{\phi}$. Then, we have
        \begin{align*}
            \phiof{x \cdot y} = \phiof{x} \cdot \phiof{y} = 0 \cdot 0 = 0
        \end{align*}
        where the last equality follows from the fact that $\cdot$ is bilinear. Therefore, $x \cdot y \in \pker{\phi}$, and $\pker{\phi}$ is closed under the multiplication map of $A$.
    \end{enumerate}
\end{proof}

This allows us to construct another matrix Lie algebra.

\begin{boxexample}[The Special Linear Lie Algebra]\label{Ch1:Eg:sl}
    For all $n \in \N$, consider the trace map $\operatorname{Tr} : \gl{n} \to \gl{1}$. This is a (Lie) algebra homomorphism: for all $A, B \in \gl{n}$,
    \begin{align*}
        \Tr{\brac{A, B}} = \Tr{AB - BA} = \Tr{AB} - \Tr{BA} = 0 = \brac{\Tr{A}, \Tr{B}}
    \end{align*}
    because the Lie algebra $\gl{1}$ is abelian (see \Cref{Ch1:Eg:gl1}). By \Cref{Ch1:Lemma:im_ker_subalg}, its kernel, the set of all $n \times n$ matrices of trace zero, is a Lie subalgebra of $\gl{n}$. We call this the \textbf{Special Linear Lie Algebra}, denoted $\sl{n}$.
\end{boxexample}

\begin{remark}
    In \Cref{Ch1:Eg:sl}, we have indirectly shown that
    \begin{align*}
        \pim{\liebrac} = \brac{\gl{n}, \gl{n}} \subseteq \sl{n}
    \end{align*}
    because of the unique property of the trace that $\Tr{AB} = \Tr{BA}$ for any $A, B \in \gl{n}$.
\end{remark}

The very natural relationship between associative and Lie algebra structures given by \Cref{Ch1:Lemma:CommBracket} gives us an elegant criterion for proving that a subspace is a subalgebra of a Lie algebra whose Lie bracket is the commutator of an associative bilinear map.

\begin{boxproposition}\label{Ch1:Prop:Subalg_Commbracket}
    Let $\parenth{A, \cdot_A}$ be an associative algebra and let $\parenth{B, \cdot_B}$ be a subalgebra of $A$. Denoting by $\parenth{A, \liebrac_A}$ the Lie algebra whose Lie bracket is the commutator of the multiplication map of $A$ and by $\parenth{B, \liebrac_B}$ the Lie algebra whose Lie bracket is the commutator of the multiplication map of $B$, we have $B' \leq A'$. In other words, the following diagram commutes:

    \begin{cd}
        \parenth{A, \cdot_A}
        \arrow[r] &[6em]
        \parenth{A, \liebrac_A} \\[1em]
        \parenth{B, \cdot_B}
        \arrow[r]
        \arrow[u, "\text{Associative Subalgebra}", hook] &[6em]
        \parenth{B, \liebrac_B}
        \arrow[u, "\text{Lie Subalgebra}"', hook', dashed]
        \label{Ch1:cd:Subalg_Commbracket}
    \end{cd}
\end{boxproposition}
\begin{proof}
    First, observe that $\liebrac_B = \liebrac_A\vert_B$ (ie, the Lie bracket obtained from $\cdot_B$ agrees with the one obtained from $\cdot_A$ on $B$): for all $T_1, T_2 \in B$,
    \begin{align*}
        \brac{T_1, T_2}_B = T_1 \cdot_B T_2 - T_2 \cdot_B T_1 = T_1 \cdot_A T_2 - T_2 \cdot_A T_1 = \brac{T_1, T_2}_A
    \end{align*}
    % The Lean coder in me wants to put coercion arrows everywhere... :,)
    Therefore, since $B$ is closed under $\liebrac_B$ (which, by definition, is a map from $B \times B$ to $B$), $B$ must be closed under $\liebrac_A$.
\end{proof}

This allows us to construct more examples still.

\begin{boxexample}[The Upper-Triangular Lie Algebra]
    For $n \in \N$, we define the \textbf{Upper-Triangular Lie Algebra} to be the set of all $n \times n$ upper-triangular matrices (with respect to some predetermined basis), denoted $\t{n}$. Given that the product of upper-triangular matrices is upper-triangular, $\t{n}$ forms an associative subalgebra of $\MnC$, and therefore, a Lie subalgebra of $\gl{n}$.
\end{boxexample}
