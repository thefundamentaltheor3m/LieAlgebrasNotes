\section{The Adjoint Representation and the Killing Form}

In this section, we will explore some of the properties of the adjoint representation, which is closely related to the killing form, a bilinear form on $L$ that will help us better understand its structure.

\subsection{Properties of the Adjoint Representation}

We begin with a result on the Jordan decomposition of the adjoint representation.

Recall that the Jordan decomposition of a linear map $T$ (from some $\C$-vector space to itself) involves expressing $T$ as $d + n$, where $d$ is diagonalisable, $n$ is nilpotent and $dn = nd$. In particular, $d$ and $n$ are simultaneously triangularisable\footnote{That all commuting linear maps are simultaneously triangularisable is a well-known fact from linear algebra, and we do not prove it here.}, meaning that there exists some basis, known as a Jordan basis of $T$, with respect to whichthe matrix of $T$ is upper-triangular, with all of its diagonal entries being its eigenvalues (given by $d$) and its super-diagonal entries being either $0$ or $1$ in a nilpotent manner (given by $n$).

We can show that the adjoint representation of a Lie algebra respects this decomposition. To that end, we show that it respects diagonalisability, nilpotency and commutativity.

We begin with diagonalisability.

\begin{boxlemma}\label{Ch2:Lemma:adDiagofDiag}
    Let $d \in \gl{n}$ be diagonal with respect to some basis. Then, $\ad{d}$ is diagonalisable.
\end{boxlemma}
\begin{proof}
    \sorry
\end{proof}

Recall that \Cref{Ch1:Lemma:adNilpotnetOfNilpotent} tells us precisely that the adjoint representation respects nilpotency. Finally, we can use the fact that the adjoint representation is a Lie algebra homomorphsim to show that it respects commutativity.

\begin{lemma}
    If $x, y \in \gl{n}$ are such that $xy = yx$, then $\pad{x} \pad{y} = \pad{y} \pad{x}$.
\end{lemma}
\begin{proof}
    Observe that $xy = yx \iff \brac{x, y} = 0$. By \Cref{Ch1:Prop:AdjointLieAlgHom}, we have that
    \begin{align*}
        \brac{\pad{x}, \pad{y}} = \pad{\brac{x, y}}
    \end{align*}
    Since $\ad$ is linear and $\brac{x, y} = 0$, we have that $\brac{\pad{x}, \pad{y}} = 0$, or, equivalently, that $\pad{x} \pad{y} = \pad{y} \pad{x}$.
\end{proof}



\subsection{The Killing Form}

We are now ready to define the killing form on $L$.

\begin{boxdefinition}[The Killing Form]
    The \textbf{killing form} on $L$ is the map $\kappa : L \times L \to \C$ defined by
    \begin{align}
        \kappa(x, y) = \Tr{\pad{x} \cdot \pad{y}}
    \end{align}
    where $\ad : L \to \gl{L}$ denotes the adjoint representation of $L$.
\end{boxdefinition}

\begin{boxconvention}
    For the remainder of this chapter, we will denote the killing form on $L$ by $\kappa$.
\end{boxconvention}

The basic properties of the killing form come from the following.

\begin{boxproposition}
    $\kappa$ is a symmetric, bilinear form on $L$.
\end{boxproposition}

We will not prove this proposition, as it involves checking basic facts from linear algebra. We will take it for granted going forward.

\begin{comment}
Recall that we can assocate to any symmetric, bilinear form a symmetric matrix of it applied to any basis of a vector space. Denote
\begin{align}
    B := \begin{bmatrix}
        \kof{e_1, e_1} & \cdots & \kof{e_1, e_n} \\
        \vdots & \ddots & \vdots \\
        \kof{e_n, e_1} & \cdots & \kof{e_n, e_n}
    \end{bmatrix}
\end{align}
where $\set{e_1, \ldots, e_n}$ is a basis of $L$ and $n$ is its $\C$-dimension. Indeed, we can diagonalise $B$ by orthogonalising $\set{e_1, \ldots, e_n}$ with respect to $\kappa$ using the Gram-Schmidt process. Therefore, we may assume, without loss of generality, that $B$ is diagonal.
\end{comment}

We will now prove some identities about the killing form. We will begin by stating a basic identity involving the trace.

\begin{lemma}\label{Ch2:Lemma:TraceAssoc}
    For all $A, B, C \in \gl{L}$, we have that
    \begin{align}
        \Tr{\brac{A, B}, C} = \Tr{A, \brac{B, C}}
        \label{Ch2:Eq:TraceAssoc}
    \end{align}
\end{lemma}
\begin{proof}
    The proof is a simple consequence of two facts: first, that matrix multiplication is associative, and second, that the trace of a product of two matrices is invariant under swapping them. We will leave the details to the reader.
\end{proof}

This gives us a similar identity for the killing form.
\begin{boxproposition}\label{Ch2:Prop:KillingAssoc}
    For any $A, B, C \in L$, we have that
    \begin{align}
        \kof{\brac{A, B}, C} = \kof{A, \brac{B, C}}
        \label{Ch2:Eq:KillingAssoc}
    \end{align}
\end{boxproposition}
\begin{proof}
    \sorry  % See iPad (read bottom board before top board)
\end{proof}
