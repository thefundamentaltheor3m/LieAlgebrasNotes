\section{The Killing Form}

Let $L$ be a finite-dimensional Lie algebra. In this section, we will explore some of the properties of the killing form, a bilinear form on $L$ that will help us better understand its structure.

\subsection{Preliminaries}

We begin by defining the killing form on $L$.

\begin{boxdefinition}[The Killing Form]
    The \textbf{killing form} on $L$ is the map $\kappa : L \times L \to \C$ defined by
    \begin{align}
        \kappa(x, y) = \Tr{\pad{x} \cdot \pad{y}}
    \end{align}
    where $\ad : L \to \gl{L}$ denotes the adjoint representation of $L$.
\end{boxdefinition}

\begin{boxconvention}
    For the remainder of this chapter, we will denote the killing form on $L$ by $\kappa$.
\end{boxconvention}

The basic properties of the killing form come from the following.

\begin{boxproposition}
    $\kappa$ is a symmetric, bilinear form on $L$.
\end{boxproposition}

We will not prove this proposition, as it involves checking basic facts from linear algebra. We will take it for granted going forward.

Recall that we can assocate to any symmetric, bilinear form a symmetric matrix of it applied to any basis of a vector space. Denote
\begin{align}
    B := \begin{bmatrix}
        \kof{e_1, e_1} & \cdots & \kof{e_1, e_n} \\
        \vdots & \ddots & \vdots \\
        \kof{e_n, e_1} & \cdots & \kof{e_n, e_n}
    \end{bmatrix}
\end{align}
where $\set{e_1, \ldots, e_n}$ is a basis of $L$ and $n$ is its $\C$-dimension. Indeed, we can diagonalise $B$ by orthogonalising $\set{e_1, \ldots, e_n}$ with respect to $\kappa$ using the Gram-Schmidt process. Therefore, we may assume, without loss of generality, that $B$ is diagonal.


