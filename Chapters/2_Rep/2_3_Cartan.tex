\section{Cartan's Criteria and the Structure Theorem}

In this section, we will see the usefulness of the Killing Form. Let $L$ be a Lie algebra, and denote by $\kappa$ the Killing Form on $L$.

\subsection{Preliminaries from Linear Algebra}

In this subsection, we will prove some important results from Linear Algebra that will prove useful going forward.

\begin{boxlemma}\label{Ch2:Lemma:PolyConjDiag}
    Let $V$ be a finite-dimensional $\C$-vector space, and let $x \in \gl{V}$ be a linear map with Jordan Decomposition $x = d + n$, where $d$ is diagonal and $n$ is nilpotent. Write
    \begin{align*}
        d = \begin{bmatrix}
                \lambda_1 & & \\
                & \ddots & \\
                & & \lambda_n
            \end{bmatrix}
    \end{align*}
    with respect to some Jordan basis, where $\lambda_1, \ldots, \lambda_n$ are the eigenvalues of $x$ and $n = \pdim{V}$. Define
    \begin{align*}
        \overline{d} :=
        \begin{bmatrix}
            \overline{\lambda_1} & & \\
            & \ddots & \\
            & & \overline{\lambda_n}
        \end{bmatrix}
    \end{align*}
    to be the diagonal matrix whose entries are the complex conjugates of the eigenvalues of $x$. If the $\lambda_i$-eigenspace $\Vof{d}_{\lambda_i}$ of $d$ is equal to the $\overline{\lambda_i}$-eigenspace $\Vof{\overline{d}}_{\overline{\lambda}_i}$ of $\overline{d}$ for all $i$, then there exists a polynomial $p \in \C[X]$ such that $p(n) = \overline{d}$.
\end{boxlemma}
\begin{proof}
    
    \sorry
\end{proof}

We will also remind the reader of the definition of non-degeneracy for bilinear forms.

\begin{boxdefinition}[Degeneracy of a Bilinear Form]
    Let $\parenth{\cdot, \cdot}$ be a bilinear form. We say that $\parenth{\cdot, \cdot}$ is \textbf{degenerate} if there exists some $x \neq 0$ such that $\parenth{x, y} = 0$ for all $y$.
\end{boxdefinition}
\begin{boxdefinition}[Non-Degeneracy of a Bilinear Form]
    We say the bilinear form $\parenth{\cdot, \cdot}$ is \textbf{non-degenerate} if it is not degenerate, ie, if for all $x \neq 0$, there exists a $y$ such that $\parenth{x, y} \neq 0$.
\end{boxdefinition}

Finally, we will note the following result on direct sums and non-degeneracy.

\begin{boxproposition}
    If $\parenth{V_1, \cycl{\cdot, \cdot}_1}$ and $\parenth{V_2, \cycl{\cdot, \cdot}_2}$ are non-degenerate, then so is $\parenth{V_1 \+ V_2, \cycl{\cdot, \cdot}_{1 \+ 2}}$, where $\cycl{\cdot, \cdot}_{1 \+ 2}$ is defined coordinate-wise.
\end{boxproposition}
\begin{proof}
    \sorry
\end{proof}

\subsection{Cartan's Criterion for Solvability}

In this subsection, we discuss and prove Cartan's Criterion for Solvability, also known as Cartan's First Criterion.

\begin{boxtheorem}[Cartan's First Criterion]\label{SP:Thm:CartanI}
    $L$ is solvable if and only if for all $\ell \in L$ and $\ell' \in L'$, $\kof{\ell, \ell'} = 0$, where $L'$ is the derived subalgebra of $L$.
\end{boxtheorem}

One direction is easily proven.

\begin{boxlemma}
    If $L$ solvable, then for all $\ell \in L$ and $\ell' \in L'$, $\kof{\ell, \ell'} = 0$.
\end{boxlemma}
\begin{proof}
    Assume that $L$ is solvable. Then, $\pad{L} \subset \gl{L}$ is solvable too: by the First Isomorphism Theorem, $\pad{L} \cong \quotient{L}{\pker{\ad}}$, and we know that any quotient of a solvable Lie algebra is solvable. Lie's Theorem then tells us that $\pad{L} \subseteq \t{L}$ (with respect to some basis of $L$), meaning that $\pad{L'} \subset \u{L}$ (as all the diagonal entries of the commutator of a matrix in $L$ with another matrix in $L$ are zero). \sorry % Finish!
\end{proof}

The converse is a bit more involved, and requires an intermediate proposition.

\begin{boxproposition}
    Let $K \leq \gl{n}$ be a Lie subalgebra such that the trace form $\parenth{A, B} \mapsto \Tr{AB} : K \times K \to \C$ is identically zero on $K$. Then, $K$ is solvable.
\end{boxproposition}
\begin{proof}
    We show that $K'$ is nilpotent. This would imply that $K'$ is solvable. Furthermore, we know that $\quotient{K}{K'}$ is abelian, and hence, solvable. Therefore, we would be able to conclude that $K$ is itself solvable. % Turn this into a separate result...

    
\end{proof}

We are now ready to prove the converse of \Cref{SP:Thm:CartanI}.

\begin{boxlemma}
    If for all $\ell \in L$ and $\ell' \in L'$, $\kof{\ell, \ell'} = 0$, then $L$ is solvable.
\end{boxlemma}
\begin{proof}
    The high-level idea is to show that $L'$ is nilpotent by applying Engel's Theorem.

    \sorry
\end{proof}

\subsection{Cartan's Criterion for Semi-Simplicity}

In this subsection, we discuss and prove Cartan's Criterion for Semi-Simplicity, also known as Cartan's Second Criterion.

\begin{boxtheorem}[Cartan's Second Criterion]\label{SP:Thm:CartanII}
    $L$ is semi-simple if and only if the Killing Form $\kappa$ is non-degenerate.
\end{boxtheorem}
\begin{proof}
    \begin{description}
        \item[$\parenth{\implies}$]
        Assume that $L$ is semi-simple and that $L^\perp \neq \set{0}$, where we consider the orthogonal complement with respect to the Killing Form $\kappa_L$. We know, from \Cref{Ch2:Prop:KillingIdeal}, that $L^\perp \nsg L$, and therefore, that $\kappa_{L^\perp} = \kappa_L\vert_{L^\perp}$. However, by definition of $L^\perp$,
        \begin{align*}
            \kappa_{L^\perp} = \kappa_L\vert_{L^\perp} = 0
        \end{align*}
        Therefore, \Cref{SP:Thm:CartanI} (Cartan's First Criterion) tells us that $L$ is solvable. However, this tells us that $L$ cannot be semi-simple, a contradiction. Therefore, $\kappa$ cannot be degenerate.
        
        \item[$\parenth{\impliedby}$] \sorry % No need to use Cartan I, there is a direct proof. Check recording (might not exist!)/Erdmann-Wildon.
    \end{description}
\end{proof}

\subsection{The Structure Theorem for Complex, Semi-Simple Lie Algebras}

We are now ready for the important Structure Theorem for Complex, semi-simple Lie algebras.

\begin{boxtheorem}[The Structure Theorem for Complex, Semi-Simple Lie Algebras]\label{SP:Thm:Structure}
    $L$ is semi-simple if and only if $L$ is a direct sum of finitely many semi-simple Lie algebras.
\end{boxtheorem}

\begin{proof}
    \begin{description}
        \item[$\parenth{\implies}$] 
    \end{description}
    Let $I \nsg L$ be a nonzero ideal. Pick $I$ to be minimal, in the sense that if any other ideal is properly contained in $I$, it must be zero.
    \sorry % Stuff from iPad

    % The idea is to use the stuff at the end of the first subsection about direct sums of nondegenerate bilinear forms
\end{proof}