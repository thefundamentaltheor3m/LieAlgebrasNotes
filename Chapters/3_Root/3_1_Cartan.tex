\section{Cartan Subalgebras}

Throughout this section, let $L$ denote a semi-simple Lie algebra. We begin by defining and studying semi-simple elements of $L$.

\subsection{Semi-Simple Elements}

\begin{boxdefinition}[Semi-Simplicity of Elements]
    An element $x \in L$ is \textbf{semi-simple} if it is equal to the diagonal part of its Jordan Decomposition, as in~\eqref{Ch2:Eq:GenJordanDecomp}.
\end{boxdefinition}

\begin{lemma}\label{Ch3:Lemma:ExistsNeZeroSemiSimple}
    $L$ admits a non-zero semi-simple element.
\end{lemma}
\begin{proof}
    Assume, for contradiction, that non non-zero element of $L$ is semi-simple. Then, by \Cref{Ch2:Thm:GenJordanDecomp}, for all $x \in L$, $\pad{x}$ is nilpotent. Engel's Theorem then tells us that $L$ is nilpotent, which is a contradiction because $L$ is semi-simple. % WHY?
\end{proof}

We can now define what a Cartan Subalgebra of $L$ is.

\begin{boxdefinition}[Cartan Subalgebra]
    A \textbf{Cartan subalgebra} of $L$ is a Lie subalgebra $H \leq L$ such that
    \begin{enumerate}[label = \normalfont\arabic*., noitemsep]
        \item Every element of $H$ is semi-simple.
        \item $H$ is abelian.
        \item $H$ is the maximal subalgebra of $L$ (with respect to inclusion) that satisfies the above two properties.
    \end{enumerate}
\end{boxdefinition}

We will prove important properties of Cartan subalgebras in the next subsection.

\subsection{Properties of Cartan Subalgebras}

First of all, we need to make sure what we are doing is sensible.

\begin{lemma}
    $L$ admits a unique Cartan subalgebra.
\end{lemma}
\begin{proof}
    The zero subalgebra satisfies the necessary properties. The unieuqeness follows from the  maximality condition.
\end{proof}

Now, we show that what we are studying is non-trivial.

\begin{boxproposition}
    Let $H$ be the Cartan subalgebra of $L$. Then, $H \neq 0$.
\end{boxproposition}
\begin{proof}
    By \Cref{Ch3:Lemma:ExistsNeZeroSemiSimple}, $L$ admits a non-zero semi-simple element. Then, $\Span{x}$ is abelian. Furthermore, any element of it is semi-simple, as $\ad$ is linear. Therefore, 
\end{proof}
